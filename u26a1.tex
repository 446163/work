\documentclass{article}

\usepackage{amssymb}
\usepackage{listings}
\usepackage{amsmath}
\usepackage{float} 
\usepackage[a4paper]{geometry} 

\begin{document}
\title{Unit 26 | Mathematics for IT Practitioners \\ \vspace{1.5cm} Assignment 1 | Matrix operations}
\author{Daniel Easteal}
\date{January 2017}
\maketitle
\newpage
\tableofcontents
\newpage
\section{Introduction}
In this assignment I will demonstrate that I know how to do matrix calculations as well as know how they work in real life in computers with graphics. 
\section{Definition of matrices}
In this initial section I have just recorded what matrices I will be using throughout the whole assignment for reference later on and so that I do not need to write them out for each question.
\begin{equation*}
	\begin{split}
	&M =
	\begin{pmatrix}
		3 & -1\\
		4 & 2
	\end{pmatrix}
	\\
	&N =
	\begin{pmatrix}
		4 & 3\\
		-3 & -1
	\end{pmatrix}
	\\
	&P =
	\begin{pmatrix}
		1 & 3 & 5\\
		-1 & 2 & 4\\
		-3 & 4 & 3
	\end{pmatrix}
\end{split}
\qquad
\qquad
\qquad
\begin{split}
	&Q =
	\begin{pmatrix}
		2 & 3 & 3\\
		4 & 4 & -2\\
		3 & -4 & 8
	\end{pmatrix}
	\\
	&R = 
	\begin{pmatrix}
		9 & 2 & 6\\
		12 & -4 & 7
	\end{pmatrix}
	\\
	&S =
	\begin{pmatrix}
		-6 & 3\\
		-3 & -2\\
		-6 & 6
	\end{pmatrix}
\end{split}
\end{equation*}
\section{P1 - Demonstrate how matrices can be used to represent data}
In this initial section I will go through the two main properties that every matrix has matrix for both describing it and for identifying the sections within it. These are the order and the index
\subsection{Order}
The order of a matrix is a measure and state the size of a matrix by saying the size of each size. For example, here is a matrix:
\[
	\begin{pmatrix}
		1 & 2 & 3\\
		4 & 5 & 6
	\end{pmatrix}
\]
As you can see, this matrix has dimensions of 2 vertically and 3 horizontally you could then write this as having dimensions of 2 by 3 (2x3) and there you have the order of the matrix. One thing to note is that the order is the wrong way around as you read vertical then horizontal. If you wrote the last matrix as having an order of 2 by 3 but that would produce the matrix of:
\[
	\begin{pmatrix}
		1 & 2\\
		3 & 4\\
		5 & 6
	\end{pmatrix}
\]
So you just need to be careful about how you do it and always do it vertically by horizontally the opposite from yo would read a graph.
\subsection{Index}
It addition to the order, another trait that is used for matrices is the index. The index is used to locate a specific cell or data pace within a matrix. The way that this works is that you will give the cell in the top right the denotion of 11 and the cell to the that gets the denotion 12 as you move to the right and so the second digit increases by one. If you were then to move down a cell then the index would be 22 because as you move down, the first number increases. One further thing to note is the notation that is used for this is you have the name of the matrix and then the index is placed after the name as a subscript.
\[
	M = 
	\begin{bmatrix}
		M_{11} & M_{12} & M_{13} & \cdots & M_{1n}\\
		M_{21} & M_{22} & M_{23} & \cdots & M_{2n}\\
		M_{31} & M_{32} & M_{33} & \cdots & M_{3n}\\
		\cdots & \cdots & \cdots & \ddots & \ddots\\
		M_{n1} & M_{n2} & M_{n3} & \ddots & M_{nn}
	\end{bmatrix}
	\begin{center}
		\small{ \textbf{An example M matrix with the indexes labeled}}
	\end{center}
\]
\subsection{Examples of uses in IT}
matrices have many uses within IT and all computers will use them on a daily basis. A main example would be their use in openGL and how computers generate and render graphics. The very basic way that it works is that the matrices are used to apply certain transformations to the graphics, like rotating and moving the shapes around in space. For example the following matrix could be used to translate (move) an object through 3D space:
\[
	\begin{pmatrix}
		0 & 0 & 0 & 2\\
	    0 & 1 & 0 & 6\\
	    0 & 0 & 1 & 3\\
	    0 & 0 & 0 & 1
	\end{pmatrix}
\]
This will then move the object in space through 2 in the x, 6 in the y and 3 in the z direction.
\section{P2 - perform addition, subtraction and scalar multiplication}
\subsection{Addition}
To add 2 matrices together you simply add each element in the first matrix to the corresponding number in the second matrix. The general case is shown below:
\begin{equation*}
	\begin{pmatrix}
		a & b & c\\
		d & e & f\\
		g & h & i
	\end{pmatrix}
	+
	\begin{pmatrix}
		A & D & G\\
		B & E & H\\
		C & F & I
	\end{pmatrix}
	=
	\begin{pmatrix}
		a+A & b+D & c+G\\
		d+B & e+E & f+H\\
		g+C & h+F & i+I\\
	\end{pmatrix}
\end{equation*}
From this general case, we just substitute in some numbers in to correct matrices that we need to add, and this is the flowing  result:
\begin{equation*}
	\begin{pmatrix}
		1 & 2 & 3\\
		4 & 5 & 6\\
		7 & 8 & 9
	\end{pmatrix}
	+
	\begin{pmatrix}
		1 & 4 & 7\\
		2 & 5 & 8\\
		3 & 6 & 9
	\end{pmatrix}
	=
	\begin{pmatrix}
		1+1 & 4+2 & 7+3\\
		2+4 & 5+5 & 8+6\\
		3+7 & 6+8 & 9+9
	\end{pmatrix}
	\Rightarrow
	\begin{pmatrix}
		2 & 6 & 10\\
		6 & 10 & 14\\
		10 & 14 & 18
	\end{pmatrix}
\end{equation*}
\subsection{Subtraction}
In a similar way to addition, to subtract 2 matrices together you simply subtract each element in the first matrix to the corresponding number in the second matrix.
\begin{equation*}
	\begin{pmatrix}
		a & b & c\\
		d & e & f\\
		g & h & i
	\end{pmatrix}
	-
	\begin{pmatrix}
		A & D & G\\
		B & E & H\\
		C & F & I
	\end{pmatrix}
	=
	\begin{pmatrix}
		a-A & b-D & c-G\\
		d-B & e-E & f-H\\
		g-C & h-F & i-I
	\end{pmatrix}
\end{equation*}
With some random numbers inserted, an example of how this works would look like this: (again, you just substitute the numbers)
\begin{equation*}
	\begin{pmatrix}
		1 & 4 & 7\\
		2 & 5 & 8\\
		3 & 6 & 9
	\end{pmatrix}
	-
	\begin{pmatrix}
		1 & 2 & 3\\
		4 & 5 & 6\\
		7 & 8 & 9
	\end{pmatrix}
	=
	\begin{pmatrix}
		1-1 & 4-2 & 7-3\\
		2-4 & 5-5 & 8-6\\
		3-7 & 6-8 & 9-9
	\end{pmatrix}
	\Rightarrow
	\begin{pmatrix}
		0 & 2 & 4\\
		-2 & 0 & 2\\
		-4 & -2 & 0
	\end{pmatrix}
\end{equation*}
\subsection{Scalar Multiplication}
Scalar multiplication is the simple to brother multiplication as it just consists of standard multiplication with no worrying about formatting. This is used for when you have a whole matrix multiplied by a single number. All you have to do is multiply each element in the matrix by the number, and that is it. Here is the general case for this:
\begin{equation*}
	n
	\begin{pmatrix}
		A & B & C\\
		D & E & F\\
		H & I & J
	\end{pmatrix}
	=
	\begin{pmatrix}
		An & Bn & Cn\\
		Dn & En & Fn\\
		Gn & Hn & In
	\end{pmatrix}
\end{equation*}
From here we can just add in some numbers so that you can see how it all woks out:
\begin{equation*}
	2.5
	\begin{pmatrix}
		3 & 1 & 4\\
		1.5 & 9 & 2.6\\
		5 & 3.5 & 8
	\end{pmatrix}
	=
	\begin{pmatrix}
		3*2.5 & 1*2.5 & 4*2.5\\
		1.5*2.5 & 9*2.5 & 2.6*2.5\\
		5*2.5 & 3.5*2.5 & 8*2.5
	\end{pmatrix}
	\Rightarrow
	\begin{pmatrix}
		7.5 & 2.5 & 10\\
		3.75 & 22.5 & 6.5\\
		12.5 & 8.75 & 20
	\end{pmatrix}
\end{equation*}
Now that we know how to do, addition subtraction and multiplication of matrices I can now carry on and do the questions that are at hand, and they are as follows:
\subsection{Questions}
For the assignment we are told to do a total of 5 addition, subtraction and scalar questions, and I will now go through them showing my calculations as they would be in the examples above. 
\subsubsection{M + N}
\[
M + N \Rightarrow
	\begin{pmatrix}
		3 & -1\\
		4 & 2
	\end{pmatrix}
	+
	\begin{pmatrix}
		4 & 3\\
		-3 & -1
	\end{pmatrix}
	=
	\begin{pmatrix}
		3+4 & -1+3\\
		4+-3 & 2+-1
	\end{pmatrix}
	\Rightarrow
	\begin{pmatrix}
		7 & 2\\
		1 & 1\\
	\end{pmatrix}
\]
\begin{center}\vspace{0.5cm}$\therefore M+N=\begin{pmatrix} 7 & 2\\ 1 & 1\end{pmatrix}$\end{center}

\subsubsection{P + Q}
\[
P + Q \Rightarrow
	\begin{pmatrix}
		1 & 3 & 5\\
		-1 & 2 & 4\\
		-3 & 4 & 3
	\end{pmatrix}
	+
	\begin{pmatrix}
		2 & 3 & 3\\
		4 & 4 & -2\\
		3 & -4 & 8
	\end{pmatrix}
\]
\[
	=
	\begin{pmatrix}
		1+2 & 3+3 & 5+3\\
		-1+4 & 2+4 & 4+-2\\
		-3+3 & 4+-4 & 3+8
	\end{pmatrix}
	\Rightarrow
	\begin{pmatrix}
		3 & 6 & 8\\
		3 & 6 & 2\\
		0 & 0 & 11 
	\end{pmatrix}
\]
\begin{center}\vspace{0.5cm}$\therefore P+Q=\begin{pmatrix} 3 & 6 & 8\\3 & 6 & 2\\0 & 0 & 11\end{pmatrix}$\end{center}

\subsubsection{M - N}
\[
M - N \Rightarrow
	\begin{pmatrix}
		3 & -1\\
		4 & 2
	\end{pmatrix}
	-
	\begin{pmatrix}
		4 & 3\\
		-3 & -1
	\end{pmatrix}
	=
	\begin{pmatrix}
		3-4 & -1-3\\
		4--3 & 2--1
	\end{pmatrix}
	\Rightarrow
	\begin{pmatrix}
		1 & -4\\
		7 & 3\\
	\end{pmatrix}
\]
\begin{center}\vspace{0.5cm}$\therefore M+N=\begin{pmatrix} 1 & -4\\ 7 & 3\end{pmatrix}$\end{center}

\subsubsection{3P}
\[
	3P \Rightarrow 3
	\begin{pmatrix}
		1 & 3 & 5\\
		-1 & 2 & 4\\
		-3 & 4 & 3
	\end{pmatrix}
	=
	\begin{pmatrix}
		3(1) & 3(3) & 3(5)\\
		3(-1) & 3(2) & 3(4)\\
		3(-3) & 3(4) & 3(3)
	\end{pmatrix}
	\Rightarrow
	\begin{pmatrix}
		3 & 9 & 15\\
		-3 & 6 & 12\\
		-9 & 12 & 9
	\end{pmatrix}
\]
\begin{center}\vspace{0.5cm}$\therefore 3P=\begin{pmatrix} 3 & 9 & 15\\ -3 & 6 & 12\\ -9 & 12 & 9\end{pmatrix}$\end{center}

\subsubsection{3P - 2Q}

For this question we will first need to calculate what 2Q is, after this I can then use my value for 3P that I found in the previous section to calculate the actual answer.
\[
	2Q \Rightarrow 2
	\begin{pmatrix}
		2 & 3 & 3\\
		4 & 4 & -2\\
		3 & -4 & 8
	\end{pmatrix}
	=
	\begin{pmatrix}
		2(2) & 2(3) & 2(3)\\
		2(4) & 2(4) & 2(-2)\\
		2(3) & 2(-4) & 2(8)
	\end{pmatrix}
	\Rightarrow
	\begin{pmatrix}
		4 & 6 & 6\\
		8 & 8 & -4\\
		6 & -8 & 16
	\end{pmatrix}
\]
\begin{center}\vspace{0.5cm}$\therefore 2Q=\begin{pmatrix} 4 & 6 & 6\\ 8 & 8 & -4\\ 6 & -8 & 16\end{pmatrix}$\end{center}
\[
	3P-2Q
	\Rightarrow
	\begin{pmatrix}
		3 & 9 & 15\\
		-3 & 6 & 12\\
		-9 & 12 & 9
	\end{pmatrix}
	-
	\begin{pmatrix}
		4 & 6 & 6\\
		8 & 8 & -4\\
		6 & -8 & 16
	\end{pmatrix}
	=
	\begin{pmatrix}
		3-4 & 9-6 & 15-6\\
		-3-8 & 6-8 & 12--4\\
		-9-6 & 12--8 & 9-16
	\end{pmatrix}
\]
\[
	\Rightarrow
	\begin{pmatrix}
		-1 & 3 & 9\\
		-11 & -2 & 16\\
		-15 & 20 & -7
	\end{pmatrix}
\]
\begin{center}\vspace{0.5cm}$\therefore 3P-2Q=\begin{pmatrix} -1 & 3 & 9\\ -11 & -2 & 16\\ -15 & 20 & -7\end{pmatrix}$\end{center}

\section{P3 - Multiply two matricies}
\subsection{Multiplication}

In order to multiply two matrices you must first ensure that the width of the first matrix is the same as the height of the second, if this is not the case then the multiplication cannot happen between them.
\[
\bordermatrix{~ & ~ & ~ & ~ \cr \rightarrow & 1 & 4 & 7 \cr ~ & 2 & 5 & 8 \cr}
\bordermatrix{~ & \downarrow & ~ \cr ~ & 4 & 7 \cr ~ & 5 & 8}
= undefined
\]
\begin{center} \small{ \textbf{The number of items indicated by the arrows do not match.\\}} \end{center}
If the multiplication can occur then you will have to add the values of each number in the first matrix row multiplied by its corresponding number in the second matrix column, this result would then go into the overlapping section for example if you just multiplied the first row by the first column then the result will go into the first row and first column of the answer matrix.

\textit{\textbf{Please note that the answer matrix may be smaller than the initial matrices.}}
\begin{equation*}
	\begin{pmatrix}
		3 & 5 & 1\\
		2 & 3 & 4
	\end{pmatrix}
	\begin{pmatrix}
		9 & 2\\
		6 & 1\\
		10 & 3
	\end{pmatrix}
	=
	\begin{pmatrix}
		(3\times9)+(5\times6)+(1\times10) & (3\times2)+(5\times1)+(1\times3)\\
		(2\times9)+(3\times6)+(4\times10) & (2\times2)+(3\times1)+(4\times3)
	\end{pmatrix}
	\Rightarrow
	\begin{pmatrix}
		67 & 14\\
		76 & 19
	\end{pmatrix}
\end{equation*}
\subsection{Questions}
\subsubsection{MN}
\[
	\begin{pmatrix}
		3 & -1\\
		4 & 2
	\end{pmatrix}
	\begin{pmatrix}
		4 & 3\\
		-3 & -1
	\end{pmatrix}
	=
	\begin{pmatrix}
		(3\times4)+(-1\times-3) & (3\times3)+(-1\times-1)\\
		(4\times4)+(2\times-3) & (4\times3)+(2\times-1)
	\end{pmatrix}
\]
\[
	\Rightarrow
	\begin{pmatrix}
		12+3 & 9+1\\
		16+-6 & 12+-2
	\end{pmatrix}
	= 
	\begin{pmatrix}
		15 & 10\\
		10 & 10
	\end{pmatrix}
\]
\begin{center}\vspace{0.5cm}$\therefore MN=\begin{pmatrix} 15 & 10\\ 10 & 10\end{pmatrix}$\end{center}
\subsubsection{PQ}
\[
	\begin{pmatrix}
		1 & 3 & 5\\
		-1 & 2 & 4\\
		-3 & 4 & 3
	\end{pmatrix}
	\begin{pmatrix}
		2 & 3 & 3\\
		4 & 4 & -2\\
		3 & -4 & 8
	\end{pmatrix}
	=
\]
\[
	\begin{pmatrix}
		(1\times2)+(3\times4)+(5\times3) & (1\times3)+(3\times4)+(5\times-4) & (1\times3)+(3\times-2)+(5\times8)\\
		(-1\times2)+(2\times4)+(4\times3) & (-1\times3)+(2\times4)+(4\times-4) & (-1\times3)+(2\times-2)+(4\times8)\\
		(-3\times2)+(4\times4)+(3\times3) & (-3\times3)+(4\times4)+(3\times-4) & (-3\times3)+(4\times-2)+(3\times8)\\
	\end{pmatrix}
\]
\[
	\Rightarrow
	\begin{pmatrix}
		2+12+15 & 3+12+-20 & 3+-6+40\\
		-2+8+12 & -3+8+-16 & -3+-4+32\\
		-6+16+9 & -9+16+-12 & -9+-8+24
	\end{pmatrix}
	=
	\begin{pmatrix}
		29 & -5 & 37\\
		18 & -11 & 25\\
		19 & -5 & 7
	\end{pmatrix}
\]
\begin{center}\vspace{0.5cm}$\therefore
	PQ=
	\begin{pmatrix}
		29 & -5 & 37\\
		18 & -11 & 25\\
		19 & -5 & 7
	\end{pmatrix}
$\end{center}
\subsubsection{RS}
\[
	\begin{pmatrix}
		9 & 2 & 6\\
		12 & -4 & 7
	\end{pmatrix}
	\begin{pmatrix}
		-6 & 3\\
		-3 & -2\\
		-6 & 6
	\end{pmatrix}
	=
\]
\[
	\begin{pmatrix}
		(9\times-6)+(2\times-3)+(6\times-6) & (9\times3)+(2\times-2)+(6\times6)\\
		(12\times-6)+(-4\times-3)+(7\times-6) & (12\times3)+(-4\times-2)+(7\times6)\\
	\end{pmatrix}
\]
\[
	\Rightarrow
	\begin{pmatrix}
		-54+-6+-36 & 27+-4+36\\
		-72+12+-42 & 36+8+42
	\end{pmatrix}
	=
	\begin{pmatrix}
		-96 & 59\\
		-102 & 86
	\end{pmatrix}
\]
\begin{center}\vspace{0.5cm}$\therefore
	RS=
	\begin{pmatrix}
		-96 & 59\\
		-102 & 86
	\end{pmatrix}
$\end{center}
\subsubsection{SR}
\[
	\begin{pmatrix}
		-6 & 3\\
		-3 & -2\\
		-6 & 6
	\end{pmatrix}
	\begin{pmatrix}
		9 & 2 & 6\\
		12 & -4 & 7
	\end{pmatrix}
	=
\]
\[
	\begin{pmatrix}
		(-6\times9)+(3\times12) & (-6\times2)+(3\times-4) & (-6\times6)+(3\times7)\\
		(-3\times9)+(-2\times12) & (-3\times2)+(-2\times-4) & (-3\times6)+(-2\times7)\\
		(-6\times9)+(6\times12) & (-6\times2)+(6\times-4) & (-6\times6)+(6\times7)
	\end{pmatrix}

\]
\[
	\Rightarrow
	\begin{pmatrix}
		-54+48 & -12+-12 & -36+21\\
		-27+-24 & -6+8 & -18+-14\\
		-54+72 & -12+-24 & -36+42
	\end{pmatrix}
	=
	\begin{pmatrix}
		-18 & -24 & -15\\
		-51 & 2 & -32\\
		18 & -36 & 6
	\end{pmatrix}
\]
\begin{center}\vspace{0.5cm}$\therefore
	SR=
	\begin{pmatrix}
		-18 & -24 & -15\\
		-51 & 2 & -32\\
		18 & -36 & 6
	\end{pmatrix}
$\end{center}

\section{P4 - Inverse and transpose}
\subsection{Inverse}
The process of generating an inverse matrix can be quite a difficult one to follow, however with the correct streps it can be done easily. To start off with, an inverse square is one where when the original and inverse square are multiplied together they will generate the identity matrix as an answer. The identity matrix is a matrix that contains all zeros apart from a single diagonal lines of ones from the top left down to the bottom right corner. 
\vspace{0.5cm}
	\[
	\begin{bmatrix}
		1 & 0 & 0 & \cdots & 0\\
		0 & 1 & 0 & \cdots & 0\\
		0 & 0 & 1 & \cdots & 0\\
		\cdots & \cdots & \cdots & \ddots & 0\\
		0 & 0 & 0 & 0 & 1
	\end{bmatrix}
	\]
	\begin{center}
		\small{ \textbf{Fig 1 \\ The identity matrix}}
	\end{center}

There are different ways to actually calculate the inverse of a matrix and these methods can differ based on the size of the matrices as well. I will now go through the general case that can be used for finding the inverse of a 2x2 matrix and then after that I will go through the general case to a higher order matrix.
\subsubsection{2x2 Matrix Inverse}
To find the inverse of a 2x2 Matrix you must first find the determinant of the matrix and then after that you can apply the general case rule that has to do with swapping and inverting numbers to get the inverse you need. 
\paragraph{Determinant}
The determinant of a matrix is a special value that is used for calculations with the matrix like finding the inverse. To find the determinant you multiply the top left and bottom right values and then subtract the bottom left and top right values. The determinant is written as the letter of the matrix with pipes either side like it was an absolute value.
\[
	A = 
	\begin{pmatrix}
		a & b\\
		c & d
	\end{pmatrix}
	\Rightarrow
	|A| = ad-cb	
\]
To see what is going on, here is an example with random numbers filled in:
\[
	A = 
	\begin{pmatrix}
		5 & 6\\
		3 & 2
	\end{pmatrix}
	\Rightarrow
	|A| = 5(2) - 3(6) = 10-18
	\Rightarrow
	-8
\]
\begin{center}\vspace{0.5cm}$\therefore |A| = -8$\end{center}

\paragraph{Inverse formula}
Now that you have the determinant to find the inverse of the matrix you just need to swap the top left and bottom right values in the matrix, then inverse the other two values to their negatives, finally you multiply this new matrix by one over the determinant. Here is the general case formula:
\[
	A = 
	\begin{pmatrix}
		a & b\\
		c & d
	\end{pmatrix}
	\Rightarrow
	A^{-1} = \frac{1}{|A|}
	\begin{pmatrix}
		d & -b\\
		-c & a
	\end{pmatrix}
	=
	\frac{1}{ad-bc}
	\begin{pmatrix}
		d & -b\\
		-c & a
	\end{pmatrix}
\]
\[
	\vspace{0.5cm}
	\Rightarrow
	\begin{pmatrix}
		\frac{1}{ad-bc}d & -\frac{1}{ad-bc}b\\
		-\frac{1}{ad-bc}c & \frac{1}{ad-bc}a
	\end{pmatrix}
\]


Doing all the steps at once may seem quite confusing, so here is it again but done step by step so you can see what happens along the way:
\begin{align}
	A  &=
	\begin{pmatrix}
		a & b\\
		c & d
	\end{pmatrix}
	&\mbox{starting equation}
	\\
	&= \frac{1}{|A|}
	\begin{pmatrix}
		a & b\\
		c & d
	\end{pmatrix}
	&\mbox{multiply by 1 over the determinant}
	\\
	&= \frac{1}{ad-bc}
	\begin{pmatrix}
		a & b\\
		c & d
	\end{pmatrix}
	&\mbox{calculate the determinant}
	\\
	&= \frac{1}{ad-bc}
	\begin{pmatrix}
		d & b\\
		c & a
	\end{pmatrix}
	&\mbox{swap \textit{a} and \textit{b}}
	\\
	&= \frac{1}{ad-bc}
	\begin{pmatrix}
		d & -b\\
		-c & a
	\end{pmatrix}
	&\mbox{invert \textit{c} and \textit{b}}
	\\
	A^{-1}&=
	\begin{pmatrix}
		\frac{1}{ad-bc}d & -\frac{1}{ad-bc}b\\
		-\frac{1}{ad-bc}c & \frac{1}{ad-bc}a
	\end{pmatrix}
	&\mbox{Multiply out}
\end{align}
Below you will see an example of how this would work when there are actually numbers in place to see how it all works:
\begin{align*}
	A  &=
	\begin{pmatrix}
		5 & 6\\
		3 & 2
	\end{pmatrix}
	\\
	&= \frac{1}{|A|}
	\begin{pmatrix}
		5 & 6\\
		3 & 2
	\end{pmatrix}
	\\
	&= \frac{1}{5(2) - 6(3)}
	\begin{pmatrix}
		5 & 6\\
		3 & 2
	\end{pmatrix}
	\\
	&= \frac{1}{-8}
	\begin{pmatrix}
		2 & 6\\
		3 & 5
	\end{pmatrix}
	\\
	&= \frac{1}{-8}
	\begin{pmatrix}
		2 & -6\\
		-3 & 5
	\end{pmatrix}
	\\
	&=
	\begin{pmatrix}
		\frac{1}{-8}\times2 & -\frac{1}{-8}\times6\\
		-\frac{1}{-8}\times3 & \frac{1}{-8}\times5
	\end{pmatrix}
	\\
	A^{-1}&=
	\begin{pmatrix}
		-\frac{1}{4} & \frac{3}{4}\\
		\frac{3}{8} & -\frac{5}{8}
	\end{pmatrix}
\end{align*}
\subsubsection{Inverse larger matrices}
In order to get the inverse of a matrix that is larger than 2x2 we need to work out both the determinant and the cofactor matrix of the main matrix we are working with. We need to work out these as we can follow the following equation to work out the inverse:
\[
	A^{-1} = \frac{1}{|A|}(cofactor Matrix of A)
\]
As you can see, we need to work out the determinant and he cofactor matrix in order to solve the equation.
\paragraph{Large matrix determinant}
The process to get the determinant of a large matrix is not actually too difficult as there is just a simple pattern to follow that you need to apply. The way this pattern works is you add together all the different diagonals of the matrix going in one direction (assuming that the diagonals go through the 'walls' of the matrix and still count) and subtract the sum of the diagonals in the other direction. 

When I say the diagonal pattern through the 'walls' of the matrix look to Fig 2 as all the same letters will be in the same FIRST pattern:
\[
	\begin{pmatrix}
		a & b & c\\
		c & a & b\\
		b & c & a
	\end{pmatrix}
\]
\begin{center}
	\small{ \textbf{Fig 2 \\ The determinant pattern}}
\end{center}

Now I will go through the whole process in the general case so that you can see how it all comes together:
\[
	A = 
	\begin{pmatrix}
		a & b & c\\
		d & e & f\\
		g & h & i
	\end{pmatrix}
	\Rightarrow
	|A| = (aei + bfg + cdh) - (ceg + bdi + afh)
\]
with numbers in place of the letters as an example it will play out like so:
\begin{align*}
	A &= 
	\begin{pmatrix}
		1 & 2 & 3\\
		4 & 5 & 6\\
		7 & 8 & 9
	\end{pmatrix}
	\\
	|A| &= (1\cdot5\cdot9+2\cdot6\cdot7+3\cdot4\cdot8) - (3\cdot5\cdot7+2\cdot4\cdot9+1\cdot6\cdot8)
	\\
	&= (45+84+96) - (105+72+48)
	\\
	&= (225) - (225)
	\\
	|A| &= 0
\end{align*}
\paragraph{CoFactor Matrix}
The final part that I need for this assignment is the cofactor matrix. The process required to get the cofactor matrix is quite complicated and requires a lot of repetition and collating results as well as finding the determinant as well. To get the cofactor matrix for each item in the main matrix you need to temporally remove the row and column that that item is in, from here you then calculate the determinant for the remaining matrix. You will then put this determinant value in a new matrix the same size as the original in the place of the initial removed item. After you do this for all items in the main matrix you then make some of them negative based on the size of the matrix you are using, now you will have a filled new matrix that contains determinant values, this new matrix is the cofactor matrix. One thing to note here is that I will be using index notation to show the places of the vales, is works just like a co-ordinate grid with the top left being 11 and the cell to the right being 12 and then the cell below that being 22 and so on...

On the next page I will go through the process in a general case so that you can see what I mean by this and how it works. 
\newpage
\[
	A = 
	\begin{pmatrix}
		a & b & c\\
		d & e & f\\
		g & h & i
	\end{pmatrix}
\]
\[
	\begin{matrix}
				A_{11} = 
				\begin{pmatrix}
					- & - & -\\
					- & e & f\\
					- & h & i
				\end{pmatrix}
			&
				A_{12} = 
				\begin{pmatrix}
					- & - & -\\
					d & - & f\\
					g & - & i
				\end{pmatrix}
			&
				A_{13} = 
				\begin{pmatrix}
					- & - & -\\
					d & e & -\\
					g & h & -
				\end{pmatrix}
		\\
				A_{21} = 
				\begin{pmatrix}
					- & b & c\\
					- & - & -\\
					- & h & i
				\end{pmatrix}
			&
				A_{22} = 
				\begin{pmatrix}
					a & - & c\\
					- & - & -\\
					g & - & i
				\end{pmatrix}
			&
				A_{23} = 
				\begin{pmatrix}
					a & b & -\\
					- & - & -\\
					g & h & -
				\end{pmatrix}
		\\
				A_{31} = 
				\begin{pmatrix}
					- & b & c\\
					- & e & f\\
					- & - & -
				\end{pmatrix}
			&
				A_{32} = 
				\begin{pmatrix}
					a & - & c\\
					d & - & f\\
					- & - & -
				\end{pmatrix}
			&
				A_{33} = 
				\begin{pmatrix}
					a & b & -\\
					d & e & -\\
					- & - & -
				\end{pmatrix}
	\end{matrix}
\]
\[
	\begin{matrix}
		|A_{11}| = \textbf{ei-fh} & |A_{12}| = \textbf{di-fg} & |A_{13}| = \textbf{dh-eg}\\
		|A_{21}| = \textbf{bi-ch} & |A_{22}| = \textbf{ai-cg} & |A_{23}| = \textbf{ah-bg}\\
		|A_{31}| = \textbf{bf-ce} & |A_{32}| = \textbf{af-cd} & |A_{33}| = \textbf{ae-bd}
	\end{matrix}
\]
\\ 
As you can see, the answer will now be the matrix that will consist of the answers of these answers that will look like this:
\[
	\begin{pmatrix}
		ei-fh & di-fg & dh-eg\\
		bi-ch & ai-cg & ah-bg\\
		bf-ce & af-cd & ae-bd
	\end{pmatrix}
\]
From here I now just need to multiply the newest matrix by the determinant + and - matrix so that I know what the signs are. This matrix always consists of + and - that alternate starting with a + in the top left corner, in this case this matrix would look like so:
\[
	\begin{pmatrix}
		+ & - & +\\
		- & + & -\\
		+ & - & +
	\end{pmatrix}
	\begin{center}
		\small{ \textbf{Fig 3 \\ The + and - determinent pattern}}
	\end{center}
\]
	Next I just need to apply the matrix by the determinant pattern to get just one step away from the answer as we have just calculate the co-factor matrix. 
\[
	\begin{pmatrix}
		ei-fh & -(di-fg) & dh-eg\\
		-(bi-ch) & ai-cg & -(ah-bg)\\
		bf-ce & -(af-cd) & ae-bd
	\end{pmatrix}
\]
The next stage that I have to do now is to transpose this merged sign matrix before I add it into the calculation. The explanation about transposing will be explained in the next section so I wont explain it now at all:
\[
	\begin{pmatrix}
		ei-fh & -(di-fg) & dh-eg\\
		-(bi-ch) & ai-cg & -(ah-bg)\\
		bf-ce & -(af-cd) & ae-bd
	\end{pmatrix}
	\right^T
	\Rightarrow
	\begin{pmatrix}
		\ddots & -(di-fg) & dh-eg\\
		-(bi-ch) & \ddots & -(ah-bg)\\
		bf-ce & -(af-cd) & \ddots
	\end{pmatrix}
\]
\[
	\begin{pmatrix}
		\ddots & -(bi-ch) & bf-ce\\
		-(di-fg) & \ddots & -(af-cd)\\
		dh-eg & -(ah-bg) & \ddots
	\end{pmatrix}
	\Rightarrow
	\begin{pmatrix}
		ei-fh & -(bi-ch) & bf-ce\\
		-(di-fg) & ai-cg & -(af-cd)\\
		dh-eg & -(ah-bg) & ae-bd
	\end{pmatrix}

\]
	Finally all that I need to do is multiply the new matrix by the determinant that I found earlier and the inverse will be found. This will follow the initial formula that I mentioned earlier that looks like the following:
\[
	A^{-1} = \frac{1}{|A|}(\mbox{cofactor Matrix of A})
\]
And so now the equation with the numbers replaced with what they have been calculated to be will look like:
\[ 
	A^{-1} = \frac{1}{(aei + bfg + cdh) - (ceg + bdi + afh)}
	\begin{pmatrix}
		ei-fh & -(bi-ch) & bf-ce\\
		-(di-fg) & ai-cg & -(af-cd)\\
		dh-eg & -(ah-bg) & ae-bd
	\end{pmatrix}
\]

\subsection{Transpose}
Another technique that can be used for the modification of matrices is the process of transposing. This is quite simple as all it does is transpose a matrix to its swapped dimension counterpart. For example a 3x2 matrix would become a 2x3 matrix.

The notation for this is the matrix with a 't' or a 'T' to the top right like a power.
\begin{equation*}
	\begin{pmatrix}
		a & b\\
		c & d
	\end{pmatrix}
	\right^t
	Or
	\begin{pmatrix}
		a & b\\
		c & d
	\end{pmatrix}
	\right^T
\end{equation*}
However it is still slightly complicated as you cannot just rotate the whole matrix as you have to rotate each line in an opposite way to its neighbours as shown in this following example:..
\begin{equation*}
	\begin{pmatrix}
		3 & 5 & 1\\
		2 & 3 & 4
	\end{pmatrix}
	\right^T
	\Rightarrow
	\begin{pmatrix}
		3 & 2\\
		5 & 3\\
		1 & 4
	\end{pmatrix}
\end{equation*}
It may help if you think that the top left and bottom right corners are locked in place during this procedure, however this will only really work when you have a matrix that is 2xn as there is a more complicated procedure other wise.
\\
The method that you actually use for a matrix of any size is actually very easy and you just need to remember the pattern although it may get a bit confusing. The pattern that you have to remember is that you keep the numbers that are in a diagonal line from the top left to the bottom right (a 45 degree lime) in the same place, and you rotate all the other numbers to the place that is on the other side of the line like a mirror. below you can see the general case for this:
\begin{equation*}
	\begin{pmatrix}
		a & b & c\\
		d & e & f\\
		g & h & i
	\end{pmatrix}
	\right^T
	\Rightarrow
	\begin{pmatrix}
		\ddots & b & c\\
		d & \ddots & f\\
		g & h & \ddots
	\end{pmatrix}
	\mbox{flip along the line}
	\Rightarrow
	\begin{pmatrix}
		\ddots & d & g\\
		b & \ddots & h\\
		c & f & \ddots
	\end{pmatrix}
	\Rightarrow
	\begin{pmatrix}
		a & d & g\\
		b & e & h\\
		c & f & i
	\end{pmatrix}
\end{equation*}
\begin{center}\vspace{0.5cm}$\therefore
	\begin{pmatrix}
		a & b & c\\
		d & e & f\\
		g & h & i
	\end{pmatrix}
	\right^T
	=
	\begin{pmatrix}
		a & d & g\\
		b & e & h\\
		c & f & i
	\end{pmatrix}
$\end{center}
now for a worked example, the process is exactly the same for any size of matrix so is is very simple:
\begin{equation*}
	\begin{pmatrix}
		40 & 9 & 6\\
		23 & 42 & 7\\
		18 & 98 & 19\\
		37 & 21 & 1
	\end{pmatrix}
	\right^T
	\Rightarrow
	\begin{pmatrix}
		\ddots & 9 & 6\\
		23 & \ddots & 7\\
		19 & 98 & \ddots\\
		37 & 21 & 1
	\end{pmatrix}
	\mbox{flip along the line}
	\Rightarrow
	\begin{pmatrix}
		\ddots & 32 & 19 & 37\\
		9 & \ddots & 98 & 21\\
		6 & 7 & \ddots & 1
	\end{pmatrix}
	\Rightarrow
	\begin{pmatrix}
		40 & 32 & 19 & 37\\
		9 & 42 & 98 & 21\\
		6 & 7 & 19 & 1
	\end{pmatrix}
\end{equation*}
\subsection{Questions}
\subsubsection{$M^{-1}$}
\begin{align*}
	M  &=
	\begin{pmatrix}
		3 & -1\\
		4 & 2
	\end{pmatrix}
	\\
	&= \frac{1}{|A|}
	\begin{pmatrix}
		3 & -1\\
		4 & 2
	\end{pmatrix}
	\\
	&= \frac{1}{3(2) - -1(4)}
	\begin{pmatrix}
		3 & -1\\
		4 & 2
	\end{pmatrix}
	\\
	&= \frac{1}{10}
	\begin{pmatrix}
		2 & -1\\
		4 & 3
	\end{pmatrix}
	\\
	&= \frac{1}{10}
	\begin{pmatrix}
		2 & 1\\
		-4 & 3
	\end{pmatrix}
	\\
	&=
	\begin{pmatrix}
		\frac{1}{10}\times2 & \frac{1}{10}\times1\\
		\frac{1}{10}\times-4 & \frac{1}{10}\times3
	\end{pmatrix}
	\\
	M^{-1}&=
	\begin{pmatrix}
		\frac{1}{5} & \frac{1}{10}\\
		-\frac{2}{5} & \frac{3}{10}
	\end{pmatrix}
\end{align*}
\subsubsection{$N^{-1}$}
\begin{align*}
	N  &=
	\begin{pmatrix}
		4 & 3\\
		-3 & -1
	\end{pmatrix}
	\\
	&= \frac{1}{|A|}
	\begin{pmatrix}
		4 & 3\\
		-3 & -1
	\end{pmatrix}
	\\
	&= \frac{1}{4(-1) - 3(-3)}
	\begin{pmatrix}
		4 & 3\\
		-3 & -1
	\end{pmatrix}
	\\
	&= \frac{1}{5}
	\begin{pmatrix}
		-1 & 3\\
		-3 & 4
	\end{pmatrix}
	\\
	&= \frac{1}{5}
	\begin{pmatrix}
		-1 & -3\\
	 	3 & 4
	\end{pmatrix}
	\\
	&=
	\begin{pmatrix}
		\frac{1}{5}\times-1 & \frac{1}{5}\times-3\\
		\frac{1}{5}\times3 & \frac{1}{5}\times4
	\end{pmatrix}
	\\
	N^{-1}&=
	\begin{pmatrix}
		-\frac{1}{5} & -\frac{3}{5}\\
		-\frac{3}{5} & \frac{4}{5}
	\end{pmatrix}
\end{align*}
\subsubsection{$P^{-1}$}
\begin{align*}
	P &= 
	\begin{pmatrix}
		1 & 3 & 5\\
		-1 & 2 & 4\\
		-3 & 4 & 3
	\end{pmatrix}
	\\
	\Rightarrow
	|P| &= 
	((1\times2\times3)+(3\times4\times-3)+(5\times-1\times4)) - ((5\times2\times-3)+(3\times-1\times3)+(1\times4\times4))
	\\
	&=
	(6+-36+-20)-(-30+-9+16)
	\\
	&=
	-50--23
	\Rightarrow
	\qquad
	\qquad
	\qquad
	\qquad
	\qquad
	\qquad
	\therefore |P| = -27
\end{align*}
\[
P =
	\begin{pmatrix}
		1 & 3 & 5\\
		-1 & 2 & 4\\
		-3 & 4 & 3
	\end{pmatrix}
\]
\[
	\begin{matrix}
				A_{11} = 
				\begin{pmatrix}
					- & - & -\\
					- & 2 & 4\\
					- & 4 & 3
				\end{pmatrix}
			&
				A_{12} = 
				\begin{pmatrix}
					- & - & -\\
					-1 & - & 4\\
					-3 & - & 3
				\end{pmatrix}
			&
				A_{13} = 
				\begin{pmatrix}
					- & - & -\\
					-1 & 2 & -\\
					-3 & 4 & -
				\end{pmatrix}
		\\
				A_{21} = 
				\begin{pmatrix}
					- & 3 & 5\\
					- & - & -\\
					- & 4 & 3
				\end{pmatrix}
			&
				A_{22} = 
				\begin{pmatrix}
					1 & - & 5\\
					- & - & -\\
					-3 & - & 3
				\end{pmatrix}
			&
				A_{23} = 
				\begin{pmatrix}
					1 & 3 & -\\
					- & - & -\\
					-3 & 4 & -
				\end{pmatrix}
		\\
				A_{31} = 
				\begin{pmatrix}
					- & 3 & 5\\
					- & 2 & 4\\
					- & - & -
				\end{pmatrix}
			&
				A_{32} = 
				\begin{pmatrix}
					1 & - & 5\\
					-1 & - & 4\\
					- & - & -
				\end{pmatrix}
			&
				A_{33} = 
				\begin{pmatrix}
					1 & 3 & -\\
					-1 & 2 & -\\
					- & - & -
				\end{pmatrix}
	\end{matrix}
\]
\[
	\begin{matrix}
		|A_{11}| = (2\times3)-(4\times4) & |A_{12}| = (-1\times3)-(4\times-3) & |A_{13}| = (-1\times4)-(2\times-3)\\
		|A_{21}| = (3\times3)-(5\times4) & |A_{22}| = (1\times3)-(5\times-3) & |A_{23}| = (1\times4)-(3\times-3)\\
		|A_{31}| = (3\times4)-(5\times2) & |A_{32}| = (1\times4)-(5\times-1) & |A_{33}| = (1\times2)-(3\times-1)
	\end{matrix}
\]
\[
	\begin{pmatrix}
		6-16 & -3--12 & -4--6\\
		9-20 & 3--15 & 4--9\\
		12-10 & 4--5 & 2--3
	\end{pmatrix}
\]
\[
	\begin{pmatrix}
		-10 & 9 & 2\\
		-11 & 18 & 13\\
		2 & 9 & 5
	\end{pmatrix}
	\Rightarrow
	\begin{pmatrix}
		\ddots & 9 & 2\\
		-11 & \ddots & 13\\
		2 & 9 & \ddots
	\end{pmatrix}
	\mbox{flip along the line}
	\Rightarrow
	\begin{pmatrix}
		\ddots & -11 & 2\\
		9 & \ddots & 9\\
		2 & 13 & \ddots
	\end{pmatrix}
	\Rightarrow
	\begin{pmatrix}
		-10 & -11 & 2\\
		9 & 18 & 9\\
		2 & 13 & 5
	\end{pmatrix}

\]
\[
	\begin{pmatrix}
		-10 & -11 & 2\\
		9 & 18 & 9\\
		2 & 13 & 5
	\end{pmatrix}
	\rightarrow
	\rightarrow
	\begin{pmatrix}
		+ & - & +\\
		- & + & -\\
		+ & - & +
	\end{pmatrix}
	\rightarrow
	\rightarrow
	\begin{pmatrix}
		-10 & 11 & 2\\
		-9 & 18 & -9\\
		2 & -13 & 5
	\end{pmatrix}
\]
\[
	P^{-1} = 
	\frac{1}{-27}
	\begin{pmatrix}
		-10 & 11 & 2\\
		-9 & 18 & -9\\
		2 & -13 & 5
	\end{pmatrix}
	=
	\begin{pmatrix}
		\frac{1}{-27}\times-10 & \frac{1}{-27}\times11 & \frac{1}{-27}\times2\\
		\frac{1}{-27}\times-9 & \frac{1}{-27}\times18 & \frac{1}{-27}\times-9\\
		\frac{1}{-27}\times2 & \frac{1}{-27}\times-13 & \frac{1}{-27}\times5
	\end{pmatrix}
	=
	\begin{pmatrix}
		\frac{10}{27} & -\frac{11}{27} & -\frac{2}{27}\\
		\frac{1}{3} & -\frac{2}{3} & \frac{1}{3}\\
		-\frac{2}{27} & \frac{13}{27} & -\frac{5}{27}
	\end{pmatrix}
\]
\begin{center}\vspace{0.5cm}$\therefore
	P^{-1}=
	\begin{pmatrix}
		\frac{10}{27} & -\frac{11}{27} & -\frac{2}{27}\\
		\frac{1}{3} & -\frac{2}{3} & \frac{1}{3}\\
		-\frac{2}{27} & \frac{13}{27} & -\frac{5}{27}
	\end{pmatrix}
$\end{center}
\subsubsection{$Q^{-1}$}
\begin{align*}
	Q &=
	\begin{pmatrix}
		2 & 3 & 3\\
		4 & 4 & -2\\
		3 & -4 & 8
	\end{pmatrix}
	\\
	\Rightarrow
	|P| &= 
	((2\times4\times8)+(3\times-2\times3)+(3\times4\times-4)) - ((3\times4\times3)+(3\times4\times8)+(2\times-2\times-4))
	\\
	&=
	(64+-18+-48)-(36+96+16)
	\\
	&=
	-2-148
	\Rightarrow
	\qquad
	\qquad
	\qquad
	\qquad
	\qquad
	\qquad
	\therefore |Q| = -150
\end{align*}
\[
Q =
	\begin{pmatrix}
		2 & 3 & 3\\
		4 & 4 & -2\\
		3 & -4 & 8
	\end{pmatrix}
\]
\[
	\begin{matrix}
				A_{11} = 
				\begin{pmatrix}
					- & - & -\\
					- & 4 & -2\\
					- & -4 & 8
				\end{pmatrix}
			&
				A_{12} = 
				\begin{pmatrix}
					- & - & -\\
					4 & - & -2\\
					3 & - & 8
				\end{pmatrix}
			&
				A_{13} = 
				\begin{pmatrix}
					- & - & -\\
					4 & 4 & -\\
					3 & -4 & -
				\end{pmatrix}
		\\
				A_{21} = 
				\begin{pmatrix}
					- & 3 & 3\\
					- & - & -\\
					- & -4 & 8
				\end{pmatrix}
			&
				A_{22} = 
				\begin{pmatrix}
					2 & - & 3\\
					- & - & -\\
					3 & - & 8
				\end{pmatrix}
			&
				A_{23} = 
				\begin{pmatrix}
					2 & 3 & -\\
					- & - & -\\
					3 & -4 & -
				\end{pmatrix}
		\\
				A_{31} = 
				\begin{pmatrix}
					- & 3 & 3\\
					- & 4 & -2\\
					- & - & -
				\end{pmatrix}
			&
				A_{32} = 
				\begin{pmatrix}
					2 & - & 3\\
					4 & - & -2\\
					- & - & -
				\end{pmatrix}
			&
				A_{33} = 
				\begin{pmatrix}
					2 & 3 & -\\
					4 & 4 & -\\
					- & - & -
				\end{pmatrix}
	\end{matrix}
\]
\[
	\begin{matrix}
		|A_{11}| = (4\times8)-(-2\times-4) & |A_{12}| = (4\times8)-(-2\times3) & |A_{13}| = (4\times-4)-(4\times3)\\
		|A_{21}| = (3\times8)-(3\times-4) & |A_{22}| = (2\times8)-(3\times3) & |A_{23}| = (2\times-4)-(3\times3)\\
		|A_{31}| = (3\times-2)-(3\times4) & |A_{32}| = (2\times-2)-(3\times4) & |A_{33}| = (2\times4)-(3\times4)
	\end{matrix}
\]
\[
	\begin{pmatrix}
		32-8 & 32--6 & -16-12\\
		24--12 & 16-9 & -8-9\\
		-6-12 & -4-12 & 8-12
	\end{pmatrix}
\]
\[
	\begin{pmatrix}
		24 & 38 & -28\\
		36 & 7 & -17\\
		-18 & -16 & -4
	\end{pmatrix}
	\Rightarrow
	\begin{pmatrix}
		\ddots & 38 & -28\\
		36 & \ddots & -17\\
		-18 & -16 & \ddots
	\end{pmatrix}
	\mbox{flip along the line}
	\Rightarrow
	\begin{pmatrix}
		\ddots & 36 & -18\\
		38 & \ddots & -16\\
		-28 & -17 & \ddots
	\end{pmatrix}
	\Rightarrow
	\begin{pmatrix}
		24 & 36 & -18\\
		38 & 7 & -16\\
		-28 & -17 & -4
	\end{pmatrix}
\]
\[
	\begin{pmatrix}
		24 & 36 & -18\\
		38 & 7 & -16\\
		-28 & -17 & -4
	\end{pmatrix}
	\rightarrow
	\rightarrow
	\begin{pmatrix}
		+ & - & +\\
		- & + & -\\
		+ & - & +
	\end{pmatrix}
	\rightarrow
	\rightarrow
	\begin{pmatrix}
		24 & -36 & -18\\
		-38 & 7 & 16\\
		-28 & 17 & -4
	\end{pmatrix}
\]
\[
	Q^{-1} = 
	\frac{1}{-150}
	\begin{pmatrix}
		24 & -36 & -18\\
		-38 & 7 & 16\\
		-28 & 17 & -4
	\end{pmatrix}
	=
	\begin{pmatrix}
		\frac{1}{-150}\times24 & \frac{1}{-150}\times-36 & \frac{1}{-150}\times-18\\
		\frac{1}{-150}\times-38 & \frac{1}{-150}\times7 & \frac{1}{-150}\times16\\
		\frac{1}{-150}\times-28 & \frac{1}{-150}\times17 & \frac{1}{-150}\times-4
	\end{pmatrix}
	=
	\begin{pmatrix}
		-\frac{4}{25} & \frac{6}{25} & \frac{3}{25}\\
		\frac{19}{75} & -\frac{7}{150} & -\frac{8}{75}\\
		\frac{14}{75} & -\frac{17}{150} & \frac{2}{75}
	\end{pmatrix}
\]
\begin{center}\vspace{0.5cm}$\therefore
	Q^{-1} = 
	\begin{pmatrix}
		-\frac{4}{25} & \frac{6}{25} & \frac{3}{25}\\
		\frac{19}{75} & -\frac{7}{150} & -\frac{8}{75}\\
		\frac{14}{75} & -\frac{17}{150} & \frac{2}{75}
	\end{pmatrix}
$\end{center}
\subsubsection{$M^{T}$}
\[
	\begin{pmatrix}
		3 & -1\\
		4 & 2
	\end{pmatrix}
	\right^T
	\Rightarrow
	\begin{pmatrix}
		\ddots & -1\\
		4 & \ddots
	\end{pmatrix}
	\mbox{flip along the line}
	\Rightarrow
	\begin{pmatrix}
		\ddots & 4\\
		-1 & \ddots
	\end{pmatrix}
	\Rightarrow
	\begin{pmatrix}
		3 & 4\\
		-1 & 2
	\end{pmatrix}
\begin{center}\vspace{0.5cm}$\therefore
	\begin{pmatrix}
		3 & -1\\
		4 & 2
	\end{pmatrix}
	\right^T
	=
	\begin{pmatrix}
		3 & 4\\
		-1 & 2
	\end{pmatrix}
$\end{center}
\]
\subsubsection{$P^{T}$}
\[
	\begin{pmatrix}
		1 & 3 & 5\\
		-1 & 2 & 4\\
		-3 & 4 & 3
	\end{pmatrix}
	\right^T
	\Rightarrow
	\begin{pmatrix}
		\ddots & 3 & 5\\
		-1 & \ddots & 4\\
		-3 & 4 & \ddots
	\end{pmatrix}
	\mbox{flip along the line}
	\Rightarrow
	\begin{pmatrix}
		\ddots & -1 & -3\\
		3 & \ddots & 4\\
		5 & 4 & \ddots
	\end{pmatrix}
	\Rightarrow
	\begin{pmatrix}
		1 & -1 & -3\\
		3 & 2 & 4\\
		5 & 4 & 3
	\end{pmatrix}
\begin{center}\vspace{0.5cm}$\therefore
	\begin{pmatrix}
		1 & 3 & 5\\
		-1 & 2 & 4\\
		-3 & 4 & 3
	\end{pmatrix}
	\right^T
	=
	\begin{pmatrix}
		1 & -1 & -3\\
		3 & 2 & 4\\
		5 & 4 & 3
	\end{pmatrix}
$\end{center}
\]
\subsubsection{$R^{T}$}
\[
	\begin{pmatrix}
		9 & 2 & 6\\
		12 & -4 & 7
	\end{pmatrix}
	\right^T
	\Rightarrow
	\begin{pmatrix}
		\ddots & 2 & 6\\
		12 & \ddots & 7
	\end{pmatrix}
	\mbox{flip along the line}
	\Rightarrow
	\begin{pmatrix}
		\ddots & 12\\
		2 & \ddots\\
		6 & 7
	\end{pmatrix}
	\Rightarrow
	\begin{pmatrix}
		9 & 12\\
		2 & -4\\
		6 & 7
	\end{pmatrix}
\]
\begin{center}\vspace{0.5cm}$\therefore
	\begin{pmatrix}
		9 & 2 & 6\\
		12 & -4 & 7
	\end{pmatrix}
	\right^T
	=
	\begin{pmatrix}
		9 & 12\\
		2 & -4\\
		6 & 7
	\end{pmatrix}
$\end{center}
\section{Simultaneous equations}
for the final section that I will go through the maths of I will be going through the process of solving simultaneous equations using matrices. This process will require a few more steps and also an understanding of how matrices work and how this relates to the equations. For this I will not be going through a general case with letters but I well do an example to explain the steps.   
\\
lets say that you have the following 2 equations:
\begin{align*}
	x+2y &=4\\
	3x-5y &=1
\end{align*}
from the knowledge that hat been explained in the previous sections, you could work out that this could be re-written as the following:
\[
	\begin{pmatrix}
		1 & 2\\
		3 & -5
	\end{pmatrix}
	\begin{pmatrix}
		x\\
		y
	\end{pmatrix}
	=
	\begin{pmatrix}
		4\\
		1
	\end{pmatrix}
\]
From here I can then reassign some temporary values to these sections so that I can explain them easier later on. 
\[
	A =
	\begin{pmatrix}
		1 & 3\\
		3 & -5
	\end{pmatrix}
	\qquad
	X =
	\begin{pmatrix}
		x\\
		y
	\end{pmatrix}
	\qquad
	B =
	\begin{pmatrix}
		4\\
		1
	\end{pmatrix}
\]
And this would therefore be able to be rearranged to this:
\[
	AX = B
\]
From here I just need to work out what the X would be and then I have the answer to what both x and y are. To do that I need to rearrange the equation so that I have X by itself and everything on the other side as I can work those out to get X. To do this I need to multiply both sides of the equations by $A^{-1}$, and this would give:
\[
	A^{-1}AX = A^{-1}B
\]
This can then be simplified as any matrix multiplied by its inverse creates the identity matrix as that is how the inverse is defined. \\ $A^{-1}A = I$ \\ In addition to this, when any matrix is multiplied by the identity matrix nothing changes with the matrix as it is like multiplying a normal number by 1. From this we then get that \\ $Xi = X$ \\ Knowing this we can then arrive to the conclusion that 
\[
	A^{-1}AX = A^{-1}B \Rightarrow X = A^{-1}B
\]
So now, all that we need to do is calculate the inverse of A (the initial matrix) and the multiply it by B (the answer matrix) and we will get the answer.
\\
With this we can now just put in the equations that we have above and work out what x and y are equal to.
\[
	A^{-1} = \frac{1}{1(-5) - 2(3)} 
	\begin{pmatrix}
		-5 & -2\\
		-3 & 1
	\end{pmatrix}
	\Rightarrow
	- \frac{1}{11}
	\begin{pmatrix}
		-5 & -2\\
		-3 & 1
	\end{pmatrix}
\]
Now that we have the inverse of A, we just need to multiply the inverted A by X and thee we are done
\[
	X = A^{-1}B = - \frac{1}{11}
	\begin{pmatrix}
		-5 & -2\\
		-3 & 1
	\end{pmatrix}
	\begin{pmatrix}
		4\\
		1
	\end{pmatrix}
	=
	- \frac{1}{11}
	\begin{pmatrix}
		-22\\
		-11
	\end{pmatrix}
	\Rightarrow
	\begin{pmatrix}
		2\\
		1
	\end{pmatrix}
\]
\begin{center}\vspace{0.5cm}$\therefore X = 
	\begin{pmatrix}
		2\\
		1
	\end{pmatrix}
	\Rightarrow
x = 2, y = 1$\end{center}
\subsection{Questions}
\subsubsection{3x + 4y = 14 \\ 2x - 7y = 11}
\[
	\begin{split}
		3x + 4y = 14 \\ 2x - 7y = 11
	\end{split}
	\qquad
	\rightarrow
	\qquad
	\begin{split}
		\begin{pmatrix}
			3 & 4\\
			2 & -7
		\end{pmatrix}
		\begin{pmatrix}
			x\\
			y
		\end{pmatrix}
		=
		\begin{pmatrix}
			14\\
			11
		\end{pmatrix}
	\end{split}
\]
From this, I can then re-arrange it to be like so:
\[
\begin{pmatrix}
		x\\
		y
	\end{pmatrix}
	=
	\begin{pmatrix}
		3 & 4\\
		2 & -7
	\end{pmatrix}
	\right^{-1}
	\begin{pmatrix}
		14\\
		11
	\end{pmatrix}
\]
First I will work out the inverse of the 3,4,2,7 matrix and then I will multiply it by the 14,11 matrix.
\begin{align*}
	A  &=
	\begin{pmatrix}
		3 & 4\\
		2 & -7
	\end{pmatrix}
	\\
	&= \frac{1}{|a|}
	\begin{pmatrix}
		3 & 4\\
		2 & -7
	\end{pmatrix}
	\\
	&= \frac{1}{3(-7) - 4(2)}
	\begin{pmatrix}
		3 & 4\\
		2 & -7
	\end{pmatrix}
	\\
	&= \frac{1}{-29}
	\begin{pmatrix}
		-7 & 4\\
		2 & 3
	\end{pmatrix}
	\\
	&= \frac{1}{-29}
	\begin{pmatrix}
		-7 & -4\\
		-2 & 3
	\end{pmatrix}
	\\
	&=
	\begin{pmatrix}
		\frac{1}{-29}\times-7 & \frac{1}{-29}\times-4\\
		\frac{1}{-29}\times-2 & \frac{1}{-29}\times3
	\end{pmatrix}
	\\
	A^{-1}&=
	\begin{pmatrix}
		\frac{7}{29} & \frac{4}{29}\\
		\frac{2}{29} & -\frac{3}{29}
	\end{pmatrix}
\end{align*}
Now I just need to multiply the inverse of a by the 14,11 matrix and I will have both x and y in a matrix
\[
	\begin{pmatrix}
		x\\
		y
	\end{pmatrix}
	=
	\begin{pmatrix}
		\frac{7}{29} & \frac{4}{29}\\
		\frac{2}{29} & -\frac{3}{29}
	\end{pmatrix}
	\begin{pmatrix}
		14\\
		11
	\end{pmatrix}
\]
\[
	\begin{pmatrix}
		\frac{7}{29} & \frac{4}{29}\\
		\frac{2}{29} & -\frac{3}{29}
	\end{pmatrix}
	\begin{pmatrix}
		14\\
		11
	\end{pmatrix}
	=
	\begin{pmatrix}
		(\frac{7}{29}\times14)+(\frac{4}{29}\times11)\\
		(\frac{2}{29}\times14)+(-\frac{3}{29}\times11)
	\end{pmatrix}
	\rightarrow
	\begin{pmatrix}
		\frac{98}{29} + \frac{44}{29}\\
		\frac{28}{29} + -\frac{33}{29}
	\end{pmatrix}
	=
	\begin{pmatrix}
		\frac{142}{29}\\
		-\frac{5}{29}
	\end{pmatrix}
\]
\begin{center}\vspace{0.5cm}$\therefore
	\begin{pmatrix}
		x\\
		y
	\end{pmatrix}
	= 
	\begin{pmatrix}
		\frac{142}{29}\\
		-\frac{5}{29}
	\end{pmatrix}
	\rightarrow
x = \frac{142}{29}, y = -\frac{5}{29}$\end{center}
\subsubsection{6x + 2y = 24 \\ 3x + 3y = 22}
\[
	\begin{split}
		6x + 2y = 24 \\ 3x + 3y = 22
	\end{split}
	\qquad
	\rightarrow
	\qquad
	\begin{split}
		\begin{pmatrix}
			6 & 2\\
			3 & 3
		\end{pmatrix}
		\begin{pmatrix}
			x\\
			y
		\end{pmatrix}
		=
		\begin{pmatrix}
			24\\
			22
		\end{pmatrix}
	\end{split}
\]
From this, I can then re-arrange it to be like so:
\[
\begin{pmatrix}
		x\\
		y
	\end{pmatrix}
	=
	\begin{pmatrix}
		6 & 2\\
		3 & 3
	\end{pmatrix}
	\right^{-1}
	\begin{pmatrix}
		24\\
		22
	\end{pmatrix}
\]
First I will work out the inverse of the 3,4,2,7 matrix and then I will multiply it by the 14,11 matrix.
\begin{align*}
	A  &=
	\begin{pmatrix}
		6 & 2\\
		3 & 3
	\end{pmatrix}
	\\
	&= \frac{1}{|a|}
	\begin{pmatrix}
		6 & 2\\
		3 & 3
	\end{pmatrix}
	\\
	&= \frac{1}{6(3) - 2(3)}
	\begin{pmatrix}
		6 & 2\\
		3 & 3
	\end{pmatrix}
	\\
	&= \frac{1}{12}
	\begin{pmatrix}
		3 & 2\\
		3 & 6
	\end{pmatrix}
	\\
	&= \frac{1}{12}
	\begin{pmatrix}
		3 & -2\\
		-3 & 6
	\end{pmatrix}
	\\
	&=
	\begin{pmatrix}
		\frac{1}{12}\times3 & \frac{1}{12}\times-2\\
		\frac{1}{12}\times-3 & \frac{1}{12}\times6
	\end{pmatrix}
	\\
	A^{-1}&=
	\begin{pmatrix}
		\frac{1}{4} & -\frac{1}{6}\\
		-\frac{1}{4} & \frac{1}{2}
	\end{pmatrix}
\end{align*}
Now I just need to multiply the inverse of a by the 14,11 matrix and I will have both x and y in a matrix
\[
	\begin{pmatrix}
		x\\
		y
	\end{pmatrix}
	=
	\begin{pmatrix}
		\frac{1}{4} & -\frac{1}{6}\\
		-\frac{1}{4} & \frac{1}{2}
	\end{pmatrix}
	\begin{pmatrix}
		24\\
		22
	\end{pmatrix}
\]
\[
	\begin{pmatrix}
		\frac{1}{4} & -\frac{1}{6}\\
		-\frac{1}{4} & \frac{1}{2}
	\end{pmatrix}
	\begin{pmatrix}
		24\\
		22
	\end{pmatrix}

	=
	\begin{pmatrix}
		(\frac{1}{4}\times24)+(-\frac{1}{6}\times22)\\
		(-\frac{1}{4}\times24)+(\frac{1}{2}\times22)
	\end{pmatrix}
	\rightarrow
	\begin{pmatrix}
		6 + -\frac{11}{3}\\
		-6 + 11
	\end{pmatrix}
	=
	\begin{pmatrix}
		\frac{7}{3}\\
		5
	\end{pmatrix}
\]
\begin{center}\vspace{0.5cm}$\therefore
	\begin{pmatrix}
		x\\
		y
	\end{pmatrix}
	= 
	\begin{pmatrix}
		\frac{7}{6}\\
		5
	\end{pmatrix}
	\rightarrow
x = -\frac{7}{6}, y = 5$\end{center}
\section{M1 - Explain the relationship between matrices and computer program variable arrays}
\subsection{Array}
In this section I will write about and explain the similarities and the differences that matrices and arrays have. To start off with, an array is a data structure that is used to store and easily retrieve information. An array is basically a one dimensional list that you can then access the elements of by referring to the name of the array and then the place in the array that you want, for example if you had an array that looked like this: A = [a,b,c,d] then you could access the data at b with A[1]. 
\\
In addition to this, due to the fact that arrays are only one dimensional, there is no such thing as two or higher dimensional array so in order to get something that looks like a 2D array you need to fill each item in the array with another array and just print it so that they line up.  
\\
Due to the fact that arrays are used in programming so much and that they are so simple the way that the information is extracted from the array is really fast so they can be used to store and retrieve information very quickly and without any delay.
\\
Furthermore, due to the fact that the array is built within a programming language there is no way to store two different types of data in an array. This means that you cannot have an array with both numbers and string values as the array can only be one. 
\\
Finally, due to the fact that arrays are a reserved space in system memory it can be very hard to change the size of and perform mathematical operations on the arrays as you will have to do it element by element and store the result in a new array and that is all very complicated and taxing on the system. Also if you wanted to change the size of an array, you would have to move all the elements one by one to a new array with a bigger defined size. 
\\
Below you can see an example of a 2D array and how they are made in the general case, in this example I am creating an array and then I am adding arrays to the arrays so that I have a 2D array.
\begin{lstlisting}
	A = [1, 2, 3, 4]
	B = [ [11, 12], [21, 22], [31, 32], [41, 42] ]
\end{lstlisting}
You can then compare this to he matrix equivalent at the end of the matrix section. 
\subsection{Matrix}
If you think about it then this is quite like an array as with an array you also store information in a grid like format and you can access the cells from the name of the array with the index as explained in the beginning.
\\
compared to an array, creating a multidimensional matrix is very easy as they are designed and set up to work like that. This is because matrices of all dimensions are used in all different ways so they are needed. 
\\
when compared to arrays, matrices are much slower to access the information and this could cause the program to be slower and perform worse. Due to this, you should try to use arrays where performance matters and use a matrix when you need the features that they need. 
\\
On the other hand, a matrix can be of any data type and contain any information in any cell in the matrix. So, if you need to store two data types then you should use an array. 
\\
Finally, performing changes to the size of a matrix is very easy as you can just add to the size of it as it is stored dynamically in system memory and so is easy to change. This also comes across very easily to mathematical operations as the dynamicness of them means that they can be operated on easily. Finally, matrix operations can be performed all at once rather than one cell at a time. 
\\
Here is the comparison from the arrays, shown in matrix format:
\[
	A = 
	\begin{pmatrix}
		1 & 2 & 3 & 4
	\end{pmatrix}
\]
\[
	B = 
	\begin{pmatrix}
		11 & 12\\
		21 & 22\\
		31 & 32\\
		41 & 42
	\end{pmatrix}
\]
\section{M2 - Apply matrix techniques to vector transformation and rotation, maps and graphs}

In this final section I will be going through how matrices are used to handle transformations and rotations of object in computer graphics, in addition to this, I will also be showing how you can represent maps and graphs with matrices. 
\\
The way that you actually translate objects in a 2D space with a matrix is that you have a 3x3 matrix that is applied to the co-ordinates of the object and then the operations that you apply to the matrix will be used to translate the object before it is rendered. 
\[
	\begin{pmatrix}
		cos(\theta) & sin(\theta) & X\\
		-sin(\theta) & cos(\theta) & Y\\
		0 & 0 & 1
	\end{pmatrix}
\]
In this matrix the X and Y are used for translation while the trig functions are used for rotation. To then move a matrix, you will then multiply the co-ordinate matrix by this matrix to get the new co-ordinate matrix:
\[
	\begin{pmatrix}
		cos(\theta) & sin(\theta) & X\\
		-sin(\theta) & cos(\theta) & Y\\
		0 & 0 & 1
	\end{pmatrix}
	\begin{pmatrix}
		a\\
		b\\
		c
	\end{pmatrix}
	=
	\begin{pmatrix}
		A\\
		B\\
		C
	\end{pmatrix}
\]
It is also important to note that the base matrix with no translations applied to it is the identity matrix and so it looks like this:
\[
	\begin{pmatrix}
		1 & 0 & 0\\
		0 & 1 & 0\\
		0 & 0 & 1
	\end{pmatrix}
\]
\subsection{Translation}
To translate an object using a matrix you replace the X and the Y values in the matrix to the amount that you want to move the object on the screen. For example, if you want to move an object 3 to the right and 1 down, then you would use the following matrix:
\[
	\begin{pmatrix}
		1 & 0 & 3\\
		0 & 1 & -1\\
		0 & 0 & 1
	\end{pmatrix}
\]
\subsection{Rotation}
In a similar fashion, if you want to rotate and object around the origin, then you would just change the $\theta$ in the trig parts with the amount that you want to rotate it from straight up being 0. For example, if you want to rotate the object 20 degrees to the right then you would just apply the following matirx:
\[
	\begin{pmatrix}
		cos(20) & sin(20) & 0\\
		-sin(20) & cos(20) & 0\\
		0 & 0 & 1
	\end{pmatrix}
\]

\subsection{Maps and Graphs}
In addition to representing translations, matrices can also represent graphs and maps, there are where there are nodes (points) and arcs (lines) that connect them. For example if you had nodes 1,2,3 and 4 and nodes 1 and 2 had two arcs connecting them and then there was one node connecting 2 to 3 and 4, then you could represent the information in a matrix like so:
\[
	\begin{pmatrix}
		0 &	2 & 0 & 0\\
		2 &	0 & 1 & 1\\
		0 &	1 & 0 & 0\\
		0 &	1 & 0 & 0
	\end{pmatrix}
\]
\section{conclusion}
In conclusion you can see that I have demonstrated that know how to do matrix calculations as well as know how they work in real life. 
\end{document}

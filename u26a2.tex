\documentclass{article}

\usepackage{amssymb}
\usepackage{listings}
\usepackage{amsmath}
\usepackage{float} 
\usepackage[a4paper]{geometry} 

\begin{document}
\title{Unit 26 | Mathematics for IT Practitioners \\ \vspace{1.5cm} Assignment 2 | Probability, Sequences, Number Systems and Statistics}
\author{Daniel Easteal}
\date{March 2017}
\maketitle
\newpage
\tableofcontents
\newpage
\section{Introduction}
test
\section{Series and Sequences and Probability}
\subsection{17th and nth term}
Find a formula for the nth term of this sequence and find the 17th term using your nth term formula. Also calculate the sum of the first 17 terms of this sequence.
\[
	-3, 1, 5, 9, 13, \dots
\]
To start off with I will need to find the difference between the numbers in the sequence that I have been given because I believe that this is a linear progression and this will get half the answer for me. Looking at the numbers, each number is 4 more than the previous one...
\[
	\overbrace{-3, 1}^{+4}	\quad	\overbrace{1, 5}^{+4}	\quad	\overbrace{5, 9}^{+4}	\quad	\overbrace{9, 13}^{+4}
\]
Now that I know the distance from each number to the next, I can then formulate the initial equation to be $$ n=4x+?$$ so now I just need to work out what the "?" will be. Now, for the first term the x value will be zero and so we would have 0 as the value, but the first value 
\subsection{Find the sum}

\[
	\sum_{r = 1}^{6} (3r - 2r^{2} + r^{3}) 
\]
\section{Number Systems}

\section{Number Systems calculations}

\section{Number Systems calculations}

\section{Recursion}

\section{Use of Number Systems}

\section{Network Planning}

\section{Number Systems calculations}

\end{document}

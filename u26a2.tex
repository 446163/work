\documentclass{article}

\usepackage{amssymb}
\usepackage{amsmath}
\usepackage{listings}
\usepackage{amsmath}
\usepackage{float} 
\usepackage[a4paper]{geometry} 

\begin{document}
\title{Unit 26 | Mathematics for IT Practitioners \\ \vspace{1.5cm} Assignment 2 | Probability, Sequences, Number Systems and Statistics}
\author{Daniel Easteal}
\date{March 2017}
\maketitle
\newpage
\tableofcontents
\newpage
\section{Introduction}
test
\section{Series and Sequences and Probability}
\subsection{-3, 1, 5, 9, 13 \dots}
\subsubsection{nth term}
Find a formula for the nth term of this sequence and find the 17th term using your nth term formula. Also calculate the sum of the first 17 terms of this sequence.
\[
	-3, 1, 5, 9, 13, \dots
\]
To start off with I will need to find the difference between the numbers in the sequence that I have been given because I believe that this is a linear progression and this will get half the answer for me. Looking at the numbers, each number is 4 more than the previous one...
\[
	\overbrace{-3, 1}^{+4}	\quad	\overbrace{1, 5}^{+4}	\quad	\overbrace{5, 9}^{+4}	\quad	\overbrace{9, 13}^{+4}
\]
Now that I know the distance from each number to the next, I can then formulate the initial equation to be $$ n=4x+?$$ so now I just need to work out what the "?" will be. Now, for the first term the x value will be one and so we would have 4 as the value, but the first value is actually -3, so we will need to adjust to this and have the equation subtract 7 from all the values so now the equation will look like this:
\[
	n=4x-7
\]
We can then test this to ensure that it works by testing the 4th value as we know the answer to that:
\[
	4(4)-7	\Rightarrow 16-7	\Rightarrow 9
\]
And this answer lines up with the answer that we are given in the question so I know that the equation is correct.  
\[
	\therefore n^{th} term = 4x-7
\]
\subsubsection{17th term}
It is very simple to find the 17th term, or really any term of the sequence once you know the nth term formula, all that you have to do is replace the "x" with the term that you want and then you have to work the equation through. For this question it would go like this: 
\[
	4x-7 \Rightarrow 4(17)-7 \Rightarrow 68-7 \Rightarrow 61
\]
\[
	\therefore 17^{th} term = 61
\]
\subsubsection{Sum of terms}
To sum the sequence that we have been given we will have to use the sum of an arithmetic sequence progression equation and this looks like this: 
\[
	\frac{n(a_{1} + a_{n})}{2}
\]
this sequence sequence is an arithmetic sequence because it advances in the series by adding numbers rather than multiplying by them. In this equation, all of the symbols start for something and I will explain them below:
\begin{align}

	n &= \mbox{represents the number of terms that you want to add, in this case it is 17}

	a_{1} &= \mbox{represents the initial number of the sequence, in this case it is -3}

	a_{n} &= \mbox{represents the last number of the sequence, in this case this is 61 as we calculated that before}

\end{align}
Now I just need to place these numbers where their letter counterpart as in the equation and then work through the equation to get the answer:
\[
	\frac{17( -3 + 61 )}{2} \Rightarrow \frac{17( 58 )}{2} \Rightarrow \frac{986}{2} \Rightarrow 493 
\]
\[
\therefore \mbox{the sum of the first 17 terms of the equation 4x-7 is 493}
\]
\subsection{81, -27, 9, -3 \dots}
\subsubsection{nth term}
As with before I will start off with working out the difference  between the values in the list that I have been given. 

\[
	\overbrace{81, -27}^{108}	\quad	\overbrace{-27, 9}^{36}	\quad	\overbrace{9, -3}^{12}
\]
As you can see here, the numbers do not match up and they are not the same, this then presents a problem because instead of adding to get to the next number we are now multiplying, this can also be shown as the difference between the difference of all the numbers is 3. What I mean by this is that the above values that are 108, 38 and 12, you will that to get to the next number you would need to divide by 3, this shows that this is a geometric progression and that I will have to use another equation.

Looking at the numbers you can see that to get to the next number in the series you would divide by -3. 3 because that is the way the differences work and then the - because the sign of the series changes every other number. Now that we know the multiple of the series we need to work out the offset. 

Again like last time the we write it out in a standard form and this would be:
\[
	a \times r^{n-1}
\]
Now that we know "r" (the common ratio) to be -1/3 we just need to put in "a" as the first number and we know that that is 81, to then it would look like this:
\[
	a \times r^{n-1} \Rightarrow 81 \times \frac{1}{3}^{(n-1)}
\]
\subsubsection{10th term}
This section is again very easy as all that I have to do is put 10 into the nth term formula and then I well get the answer that I want. This would then work through like so:
\[
	81 \times -\frac{1}{3}^{10-1} \Rightarrow 81 \times -\frac{1}{3}^{9} \Rightarrow 81 \times -\frac{1}{19683} \Rightarrow -\frac{81}{19683} \Rightarrow -\frac{1}{243}
\]
\subsubsection{Sum to the 5th element}

In order to sum a geometric series we need to know a new formula and this is the formula that we need:
\[
	S_{n} = \frac{a_{1} ( r^{n} - 1 )}{ r - 1 }
\]
As the other ones I will just insert the numbers into the formula and then go about and solve the equasion to get to the answer that we need to get to:
\[
	\frac{a_{1} ( r^{n} - 1 )}{ r - 1 } \Rightarrow \frac{81 ( -\frac{1}{3}^{5} - 1 )}{ -\frac{1}{3} - 1 } \Rightarrow \frac{81 ( -0.004115226... - 1 )}{ -1.\dot{3} } \Rightarrow \frac{81 \times -1.004115226...}{-1.\dot{3}} \Rightarrow \frac{-81.\dot{3}}{-1.\dot{3}} \Rightarrow 61
\]
\subsubsection{Sum to infinity}

In order to sum a geometric series to infinity you need to ensure that the common ration is between -1 and 1 as otherwise the sum would be infinity and not calculated. The new equation that we need to sum to infinity is as follows: 
\[
	S = \frac{a}{(1 - r)}
\]
As in the previous examples, the "a" represents the first number in the series and the "r" represents the common ration. In this case that would be 81 and -1/3 so now the formula would look like this:
\[
	S = \frac{a}{(1 - r)} \Rightarrow \frac{81}{(1 - -\frac{1}{3})} \Rightarrow \frac{81}{(1 +\frac{1}{3})} \Rightarrow \frac{81}{(1.\dot{3})} \Rightarrow 60.75  
\]
\subsection{3r - 2r^{2} + r^{3}}
In this section I have to calculate the answer to the following:
\[
	\sum_{r = 1}^{6} (3r - 2r^{2} + r^{3})
\]
The way that this works is that I have to calculate the part of the equation that is inside the brackets by replacing the "r" in this case each time from the value on the bottom to the vale on the top. Then you add all these answers and you are done. 
\[
	((3 \times 1) - ( 2 \times ( 1^{2} )) + ( 1^{3}) + ((3 \times 2) - ( 2 \times ( 2^{2} )) + ( 2^{3}) + ((3 \times 3) - ( 2 \times ( 3^{2} )) + ( 3^{3}) + ((3 \times 4) - ( 2 \times ( 4^{2} )) + ( 4^{3}) 
\]
\[
	+ ((3 \times 5) - ( 2 \times ( 5^{2} )) + ( 5^{3}) + ((3 \times 6) - ( 2 \times ( 6^{2} )) + ( 6^{3})
\]
\[
	3-2+1+6-8+16+9-18+27+12-32+64+15-50+125+18-72+216 \Rightarrow 322
\]
\[
\therefore \sum_{r = 1}^{6} (3r - 2r^{2} + r^{3}) = 322
\]
\subsection{5 balls in a bag}
There are 3 red and 2 yellow balls in a bag:
\subsubsection{yellow ball}
The probability that a yellow ball is selected is 2/5 due to the fact that there are 2 yellow balls in the bag out of a total of 5. This can be written as 0.4
\subsubsection{2 yellow balls in a row}
When there are 2 chances that need to be added up in a row then the individual chances are multiplied. For this we would then multiply the 0.4 chance from the previous question by itself as it is the same question we just have to do it twice in a row. 
\[
	0.4 * 0.4 = 0.16e
\]
\[
	\therefore \mbox{the chance of getting two yellow balls in a row is 0.16 or }\frac{2}{25}
\]
\subsubsection{Tree}
This section will be on the insert provided with the document...
\subsection{In a year group}
\subsubsection{Venn diagram}
This section will be on the insert provided with the document...
\subsubsection{Only Computer science}
To start to answer this question I first need to add up how many different students there are in total and then I can work out the probability. 
\[
	70+83+15+12+10 = 190
\]
Now that I know there are 190 students in the year I can now look at the information and see that there are 70 students that fit the category of only studying computer science and to the probability is now:
\[
	70/190 \Rightarrow 7/19 \Rightarrow (100/ 19)+7 = 12.26315789473684210526\% 
\]
\[
	\mbox{ or a probability of } 7/19 = 0.36842105263157894736  
\]
\subsubsection{Engineering}
looking at the information that I have there are 83+12 students that study Engineering in one way of another and as such there is a probability as follows for one to be randomly selected:
\[
	95 / 190 \Rightarrow 1/2 \mbox{ That is 50\% or a probability of 0.5 }
\]
\subsection{Betting game}
This section will be on the insert provided with the document...
\section{Number Systems}
\subsection{Binary, Octal, Decimal and Hexadecimal conversion}
\begin{center}
\begin{tabular}{ ||c|c c c c|| } 
 \hline
   & Denary & Binary & Octal & Hexadecimal \\
 \hline
 \hline
 a & \textbf{22} & 10110 & 26 & \textbf{16} \\ 
 \hline
 b & 11 & 1011 & \textbf{13} & B \\ 
 \hline
 c & \textbf{41} & 101001 & 51 & 29 \\ 
 \hline
 d & 20 & \textbf{10100} & 24 & 14 \\ 
 \hline
 e & 30 & 11110 & \textbf{36} & 1E \\ 
 \hline
 f & 42 & 101010 & 52 & \textbf{2A} \\
 \hline
 g & \textbf{271} & 100001111 & 417 & 10F\\ 
 \hline
 h & 44 & 101100 & 54 & 2C \\ 
 \hline
 i & 62 & 111110 & 76 & 3E \\ 
 \hline
\end{tabular}
\end{center}
\section{Number Systems calculations}
\subsection{a + f in hexadecimal}
\[
	22 + 42 = 64 \rightarrow 64_{10} = 40_{16}
\]
\[
	\therefore A+F = 40
\]
\subsection{g * f in hexadecimal}
\[
	271 * 42 = 11382 \rightarrow 11382_{10} = 2C76_{16}
\]
\[
	\therefore g * f = 2C76
\]
\subsection{f - b in hexadecimal}
\[
	42 - 11 = 31 \rightarrow  31_{10} = 1F_{16}
\]
\[
	\therefore f - b = 1F
\]
\subsection{a + e in Octal}
\[
	22 + 30 = 52 \rigtharrow 52_{10} = 64_{8}
\]
\[
	\therefore a + e = 64
\]
\subsection{e - b in Octal}
\[
	30 - 11 = 19 \rigtharrow 19_{10} = 23_{8}
\]
\[
	\therefore e - b = 23
\]
\subsection{a + d in binary}
\[
	22 + 20 = 42 \rightarrow 42_{10} = 101010_{2}
\]
\[
	\therefore a + d = 101010
\]
\subsection{a * d in binary}
\[
	22 * 20 = 440 \rightarrow 440_{10} = 110111000_{2}
\]
\[
	\therefore a * d = 110111000
\]
\subsection{h + i in binary}
\[
	44 + 62 = 106 \rightarrow 106_{10} = 1101010_{2}
\]
\[
	\therefore h + i = 1101010
\]
\subsection{h * i in hexadecimal}
\[
	44 * 62 = 2728 \rigtharrow 2728_{10} = AA8_{16}
\]
\[
	\therefore h * i = AA8
\]
\section{Data task}

\section{Recursion}
For this section I will go through and explaining how the binary search algorithm works and it fits in with the idea of recursion. Below you will see a basic format for the algorithm written in pseudo code:

\begin{verbatim}
for every item in the data set, read it into an array. 
if the array is not sorted then sort it
start:
go to the middle of the array 
is the current item what you are looking for?
....no: is the current item less then the item you are looking for?
........yes: remove all items in the array that are lower and return to start.
........no: remove all items in the array that are greater and return to start.
....yes: you are done as you  have found the item you are looking for, end the program. 
\end{verbatim}
As you can see here there are not many stages that are needed for the search algorithm and it is very simple to do. Due to this, the whole part of the program after start: goto will be run over and over again and this will lead to the item that you wand being found. I will now go through the algorithm step by step so that you know why each part is done like it is. 

\begin{verbatim}
for every item in the data set, read it into an array. 
\end{verbatim}
This line will read all of the information that you need into the program so that you can actually modify it and run the search. 

\begin{verbatim}
if the array is not sorted then sort it
\end{verbatim}
This line will sort the data in the array so that it is in an order that can be compared easily, like a cost or number. This will mean that later when you test if the data is greater or lower, you will know that you are removing only data that will not be correct. 

\begin{verbatim}
start:
\end{verbatim}
This line just marks this section in the code with a name so that I cat return to it later in the program. 

\begin{verbatim}
go to the middle of the array 
\end{verbatim}
This line will set the location of the program to look at the middle of the array of numbers, if the length of the array is odd then the program should just go to either one of the middle values as it does not really matter. 

\begin{verbatim}
is the current item what you are looking for?
\end{verbatim}
This will check if the item in the array that you are currently looking at in the program is actually the one that you want. This well then lead off to two conditions depending on if you have found the value that you are looking for or not. If you have found the item that you are looking for then you are done, but otherwise the program carries on. 

\begin{verbatim}
....no: is the current item less then the item you are looking for?
\end{verbatim}
This line will run if the item that the program is on is not the value that you want, that is what the no: part means. Now this is where the clever part of the algorithm starts. If the value of the item that you want is lower then where you are at the moment then you can remove all of the items that are greater than the current item as it is less then this one so it is not higher. This is why the array was sorted to start off with as now you can just remove the half of the array that is higher, and just like that you have halved the values to search. 

\begin{verbatim}
........yes: remove all items in the array that are lower and return to start.
\end{verbatim}
If the item that you are are on is lower than what you want then you can remove all of the items that are lower and this is what this section does. After you have removed the values then you can return to the start: marker of the program and recur it again. 

\begin{verbatim}
........no: remove all items in the array that are greater and return to start.
\end{verbatim}
If the item that you are are on is greater than what you want then you can remove all of the items that are greater and this is what this section does. After you have removed the values then you can return to the start: marker of the program and recur it again. 

\begin{verbatim}
....yes: you are done as you  have found the item you are looking for, end the program. 
\end{verbatim}
If the value that you have at the middle of the selection is the number that you are looking for then you have found it and can see what information it has that you need. Now you just get that information and save it and end the program and you are done. This is the terminal condition. 

\section{Use of Number Systems}
In this section I will go through how binary, octal and hexadecimal are used and are applied in areas of computing with how they are used and how they work. For this I will be showing how binary is used with ASCII, how hexadecimal is used with MIME and how Octal is used with Unix file permissions. 
\subsection{binary - ASCII}
One way that binary is used in computer science is in the use of ASCII. ASCII stands for the American Standard Code for Information Interchange and this is the standard that is used for how the standard set of English characters and numbers are encoded in binary on the computer and how it deals with them. With ASCII the letter "a" for example will have a value inside the computer of 1100001. So whenever the computer does anything with the letter a it will use this value of 1100001 internally. Now a cl;aver part of this is that "b" is 1100010, one more than "a" but the right most digits correlate to the numbers as 1 is 1 and 10 is 2. So for example if you had "s" the 19th letter in the alphabet then its ASCII value would be 1110011, and 19 in binary is 10011. This ASCII also extends to capital letters in a similar fashion as well, just with a different prefix and there is also all of the things that you can type on a English keyboard.  
\subsection{Hexadecimal - MIME}
To start off with, mime is a mechanism that enables the transmission of information in email format between computers that may not have the same language. Due to the fact that the whole point is that the computers have different languages, the 8 bits that where used in the previous example will not be enough and as such we need to use 2 Hexadecimal bits, in addition to this it also uses formatting to ensure that all messages fit a certain line length so that it is compatible and so on. This as you can see will use hexadecimal characters to represent the information and this is how they are used in computing. So, for example, a value of 12 will be "=0C". 
\subsection{Octal - Unix file permissions}
A way that octal is used in computing is in the UNIX and UNIX-like operating systems for setting file permissions for the users of the system. The way that the file permissions are handled in UNIX is that the file has meta data set aside for it that will store this information and there are 9 main bits for the file access permissions and a few more for the file type. The 9 main bits are separated into 3 sections that represent the access that the owner, group and anyone else has for the file. These 3 bits then represent the ability that they have to read, write and execute the file. So the bits are arranged like so
\begin{verbatim}
rwxrwxrwx
\end{verbatim}
As you can see, there are the 3 groups there and the (r)ead, (w)rite and e(x)ecute that the 3 groups have. Now the octal comes in when you want to change the value of these bits and the way that you do this is through the use of the chmod command. You give the chmod command the file that you want to edit and then you give it 3 octal numbers. The 3 octal numbers are then converted to binary and this then lines up with bits in the file and where there is a one that will be active. Lets say that you have a file called .xinitrc that you wanted to change the permissions of and you ran the command chmod .xinitrc 752, lets see what would happen:
\[
	7_{10} = 111_{2} \quad \& \quad 5_{10} = 101_{2} \quad \& \quad 2_{10} = 010_{2}
\]
\[
	\frac{\downarrow111101010\downarrow}{\mbox{rwxrwxrwx}} \Rightarrow \mbox{rwxr-x-w-}
\]
And now the permissions are changed using just octal numbers. This process is called masking. 
\section{Network Planning}
In this section I will go through and design 3 subnets for 3 different networks each of a different size.
\subsection{1000 hosts}
When designing a subnet you will need to find the power of 2 that is as low as possible while being higher than the target number of hosts. In this case the number is 1000, and the power of 2 that satisfies that requirement is $2^{10}$ as that is 1024. Now that I have this number I will need to then take a string of 32 ones and make, in this case the 10 rightmost ones 0's. This will mean that I now have the following:
\[
	11111111111111111111111111111111 \rightarrow 11111111111111111111110000000000
\]
Now I need to separate that number into four groups of 8 bits like so and then convert those 8 bet numbers into decimal to get the subnet mask and address:
\[
	\frac{\downarrow11111111\quad11111111\quad11111100\quad00000000\downarrow}{255 \qquad \qquad 255 \qquad \qquad 252 \qquad \qquad 000} \Rightarrow 255.255.252.0
\]
Now I have the subnet mask 255.255.252.0. And in this case this would be the most efficient and justified method for this network due to the fact that it allows for all of the devices that need to connect to the network to connect while taking up as few of the bits for the subnetting as possible so that more of these networks will be able to be up at a time. 
\subsection{200 hosts}
with this I will do the same process but with the number 200 this time. 
	
The lowest number that satisfies what I need is $2^{8}$ as this number is 256. Now I will have the binary number as follows:
\[
	11111111111111111111111111111111 \rightarrow 11111111111111111111111100000000
\]
This then goes to:
\[
	\frac{\downarrow11111111\quad11111111\quad11111111\quad00000000\downarrow}{255 \qquad \qquad 255 \qquad \qquad 255 \qquad \qquad 000} \Rightarrow 255.255.255.0
\]
And so the address is 255.255.255.0. And as before this is efficient for the same reason. 
\subsection{30 hosts}
with 30 hosts the best number would be $2^{5}$ and that is 32 as that fits in all the hosts. This would lead to the binary of this:
\[
	11111111111111111111111111111111 \rightarrow 11111111111111111111111111100000
\]
\[
	\frac{\downarrow 11111111 \quad 11111111 \quad 11111111 \quad 11100000 \downarrow}{255 \qquad \qquad 255 \qquad \qquad 255 \qquad \qquad 224} \Rightarrow 255.255.255.224
\]


\end{document}

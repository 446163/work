\documentclass{article}

\usepackage{amssymb}
\usepackage{listings}
\usepackage{amsmath}
\usepackage{float} 
\usepackage[a4paper]{geometry} 

\begin{document}
\title{Unit 26 | Mathematics for IT Practitioners \\ \vspace{1.5cm} Assignment 2 | Probability, Sequences, Number Systems and Statistics}
\author{Daniel Easteal}
\date{March 2017}
\maketitle
\newpage
\tableofcontents
\newpage
\section{Introduction}
test
\section{Series and Sequences and Probability}
\subsection{-3, 1, 5, 9, 13 \dots}
\subsubsection{nth term}
Find a formula for the nth term of this sequence and find the 17th term using your nth term formula. Also calculate the sum of the first 17 terms of this sequence.
\[
	-3, 1, 5, 9, 13, \dots
\]
To start off with I will need to find the difference between the numbers in the sequence that I have been given because I believe that this is a linear progression and this will get half the answer for me. Looking at the numbers, each number is 4 more than the previous one...
\[
	\overbrace{-3, 1}^{+4}	\quad	\overbrace{1, 5}^{+4}	\quad	\overbrace{5, 9}^{+4}	\quad	\overbrace{9, 13}^{+4}
\]
Now that I know the distance from each number to the next, I can then formulate the initial equation to be $$ n=4x+?$$ so now I just need to work out what the "?" will be. Now, for the first term the x value will be one and so we would have 4 as the value, but the first value is actually -3, so we will need to adjust to this and have the equation subtract 7 from all the values so now the equation will look like this:
\[
	n=4x-7
\]
We can then test this to ensure that it works by testing the 4th value as we know the answer to that:
\[
	4(4)-7	\Rightarrow 16-7	\Rightarrow 9
\]
And this answer lines up with the answer that we are given in the question so I know that the equation is correct.  
\[
	\therefore n^{th} term = 4x-7
\]
\subsubsection{17th term}
It is very simple to find the 17th term, or really any term of the sequence once you know the nth term formula, all that you have to do is replace the "x" with the term that you want and then you have to work the equation through. For this question it would go like this: 
\[
	4x-7 \Rightarrow 4(17)-7 \Rightarrow 68-7 \Rightarrow 61
\]
\[
	\therefore 17^{th} term = 61
\]
\subsubsection{Sum of terms}
To sum the sequence that we have been given we will have to use the sum of an arithmetic sequence progression equation and this looks like this: 
\[
	\frac{n(a_{1} + a_{n})}{2}
\]
this sequence sequence is an arithmetic sequence because it advances in the series by adding numbers rather than multiplying by them. In this equation, all of the symbols start for something and I will explain them below:
\begin{align}

	n &= \mbox{represents the number of terms that you want to add, in this case it is 17}

	a_{1} &= \mbox{represents the initial number of the sequence, in this case it is -3}

	a_{n} &= \mbox{represents the last number of the sequence, in this case this is 61 as we calculated that before}

\end{align}
Now I just need to place these numbers where their letter counterpart as in the equation and then work through the equation to get the answer:
\[
	\frac{17( -3 + 61 )}{2} \Rightarrow \frac{17( 58 )}{2} \Rightarrow \frac{986}{2} \Rightarrow 493 
\]
\[
\therefore \mbox{the sum of the first 17 terms of the equation 4x-7 is 493}
\]
\subsection{81, -27, 9, -3 \dots}
\subsubsection{nth term}
As with before I will start off with working out the difference  between the values in the list that I have been given. 

\[
	\overbrace{81, -27}^{108}	\quad	\overbrace{-27, 9}^{36}	\quad	\overbrace{9, -3}^{12}
\]
As you can see here, the numbers do not match up and they are not the same, this then presents a problem because instead of adding to get to the next number we are now multiplying, this can also be shown as the difference between the difference of all the numbers is 3. What I mean by this is that the above values that are 108, 38 and 12, you will that to get to the next number you would need to divide by 3, this shows that this is a geometric progression and that I will have to use another equation.

Looking at the numbers you can see that to get to the next number in the series you would divide by -3. 3 because that is the way the differences work and then the - because the sign of the series changes every other number. Now that we know the multiple of the series we need to work out the offset. 

Again like last time the we write it out in a standard form and this would be:
\[
	a \times r^{n-1}
\]
Now that we know "r" (the common ratio) to be -1/3 we just need to put in "a" as the first number and we know that that is 81, to then it would look like this:
\[
	a \times r^{n-1} \Rightarrow 81 \times \frac{1}{3}^{(n-1)}
\]
\subsubsection{10th term}
This section is again very easy as all that I have to do is put 10 into the nth term formula and then I well get the answer that I want. This would then work through like so:
\[
	81 \times -\frac{1}{3}^{10-1} \Rightarrow 81 \times -\frac{1}{3}^{9} \Rightarrow 81 \times -\frac{1}{19683} \Rightarrow -\frac{81}{19683} \Rightarrow -\frac{1}{243}
\]
\subsubsection{Sum to the 5th element}

In order to sum a geometric series we need to know a new formula and this is the formula that we need:
\[
	S_{n} = \frac{a_{1} ( r^{n} - 1 )}{ r - 1 }
\]
As the other ones I will just insert the numbers into the formula and then go about and solve the equasion to get to the answer that we need to get to:
\[
	\frac{a_{1} ( r^{n} - 1 )}{ r - 1 } \Rightarrow \frac{81 ( -\frac{1}{3}^{5} - 1 )}{ -\frac{1}{3} - 1 } \Rightarrow \frac{81 ( -0.004115226... - 1 )}{ -1.\dot{3} } \Rightarrow \frac{81 \times -1.004115226...}{-1.\dot{3}} \Rightarrow \frac{-81.\dot{3}}{-1.\dot{3}} \Rightarrow 61
\]
\subsubsection{Sum to infinity}

In order to sum a geometric series to infinity you need to ensure that the common ration is between -1 and 1 as otherwise the sum would be infinity and not calculated. The new equation that we need to sum to infinity is as follows: 
\[
	S = \frac{a}{(1 - r)}
\]
As in the previous examples, the "a" represents the first number in the series and the "r" represents the common ration. In this case that would be 81 and -1/3 so now the formula would look like this:
\[
	S = \frac{a}{(1 - r)} \Rightarrow \frac{81}{(1 - -\frac{1}{3})} \Rightarrow \frac{81}{(1 +\frac{1}{3})} \Rightarrow \frac{81}{(1.\dot{3})} \Rightarrow 60.75  
\]
\subsection{3r - 2r^{2} + r^{3}}
In this section I have to calculate the answer to the following:
\[
	\sum_{r = 1}^{6} (3r - 2r^{2} + r^{3})
\]
The way that this works is that I have to calculate the part of the equation that is inside the brackets by replacing the "r" in this case each time from the value on the bottom to the vale on the top. Then you add all these answers and you are done. 
\[
	((3 \times 1) - ( 2 \times ( 1^{2} )) + ( 1^{3}) + ((3 \times 2) - ( 2 \times ( 2^{2} )) + ( 2^{3}) + ((3 \times 3) - ( 2 \times ( 3^{2} )) + ( 3^{3}) + ((3 \times 4) - ( 2 \times ( 4^{2} )) + ( 4^{3}) 
\]
\[
	+ ((3 \times 5) - ( 2 \times ( 5^{2} )) + ( 5^{3}) + ((3 \times 6) - ( 2 \times ( 6^{2} )) + ( 6^{3})
\]
\[
	3-2+1+6-8+16+9-18+27+12-32+64+15-50+125+18-72+216 \Rightarrow 322
\]
\[
\therefore \sum_{r = 1}^{6} (3r - 2r^{2} + r^{3}) = 322
\]
\subsection{5 balls in a bag}
There are 3 red and 2 yellow balls in a bag:
\subsubsection{yellow ball}
The probability that a yellow ball is selected is 2/5 due to the fact that there are 2 yellow balls in the bag out of a total of 5. This can be written as 0.4
\subsubsection{2 yellow balls in a row}
When there are 2 chances that need to be added up in a row then the individual chances are multiplied. For this we would then multiply the 0.4 chance from the previous question by itself as it is the same question we just have to do it twice in a row. 
\[
	0.4 * 0.4 = 0.16e
\]
\[
	\therefore \mbox{the chance of getting two yellow balls in a row is 0.16 or }\frac{2}{25}
\]
\subsubsection{Tree}
This section will be on the insert provided with the document...
\subsection{In a year group}
\subsubsection{Venn diagram}
This section will be on the insert provided with the document...
\subsubsection{Only Computer science}
To start to answer this question I first need to add up how many different students there are in total and then I can work out the probability. 
\[
	70+83+15+12+10 = 190
\]
Now that I know there are 190 students in the year I can now look at the information and see that there are 70 students that fit the category of only studying computer science and to the probability is now:
\[
	70/190 \Rightarrow 7/19 \Rightarrow (100/ 19)+7 = 12.26315789473684210526\% 
\]
\[
	\mbox{ or a probability of } 7/19 = 0.36842105263157894736  
\]
\subsubsection{Engineering}
looking at the information that I have there are 83+12 students that study Engineering in one way of another and as such there is a probability as follows for one to be randomly selected:
\[
	95 / 190 \Rightarrow 1/2 \mbox{ That is 50\% or a probability of 0.5 }
\]
\subsection{Betting game}
This section will be on the insert provided with the document...
\section{Number Systems}
\subsection{Binary, Octal, Decimal and Hexadecimal conversion}
\begin{center}
\begin{tabular}{ ||c|c c c c|| } 
 \hline
   & Denary & Binary & Octal & Hexadecimal \\
 \hline
 \hline
 a & \textbf{22} & 10110 & 26 & \textbf{16} \\ 
 \hline
 b & 11 & 1011 & \textbf{13} & B \\ 
 \hline
 c & \textbf{41} & 101001 & 51 & 29 \\ 
 \hline
 d & 20 & \textbf{10100} & 24 & 14 \\ 
 \hline
 e & 30 & 11110 & \textbf{36} & 1E \\ 
 \hline
 f & 42 & 101010 & 52 & \textbf{2A} \\
 \hline
 g & \textbf{271} & 100001111 & 417 & 10F\\ 
 \hline
 h & 44 & 101100 & 54 & 2C \\ 
 \hline
 i & 62 & 111110 & 76 & 3E \\ 
 \hline
\end{tabular}
\end{center}
\section{Number Systems calculations}
\subsection{A + F}
\[
	A_{16} = 10_{10}, F_{16} = 15_{10} \therefor A+F = 25_{10} = 19_{16}
\]
\section{Data task}

\section{Recursion}

\section{Use of Number Systems}

\section{Network Planning}

\section{Number Systems calculations}

\end{document}
